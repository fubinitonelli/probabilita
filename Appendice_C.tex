\clearpage

\section*{Appendice C. Distribuzioni continue} \label{appendice-continue} %%%Chiunque tocchi: attenzione all'indentaturaaaa!!!!
\markboth{Appendice C. Distribuzioni continue}{}
\addtocounter{section}{1}
\setcounter{subsection}{0}
\setcounter{teo}{0}
\addcontentsline{toc}{section}{Appendice C. Distribuzioni continue}

%%%Chiunque tocchi: attenzione all'indentaturaaaa!!!!
Riportiamo qui una lista di alcune distribuzioni continue notevoli corredate da grafici e formule.

\needspace{7\baselineskip}
\subsection{Uniforme continua}

	$$ U_{cont}(a,b) $$

	Probabilità uniforme su un intervallo continuo $[a,b]$, con $a < b$.

	\begin{figure} [H]

		\centering

		\begin{tikzpicture}

			\begin{axis}[
			axis lines = middle,
			ylabel = $f(x)$,
			xlabel = $x$,
			width=0.57\textwidth,
			height=0.5\textwidth,
			xmin=-2,
			xmax= 5,
			ymin=-0.3,
			ymax= 1.5,
			yticklabels={,,},
			%xticklabels={,,},
			%axis equal=true
			legend style={at={(axis cs: 5,1.1)},anchor=south east, font=\tiny},
			legend cell align=left
			]

				\draw [line width=0.3mm] (axis cs:-2,0) -- (axis cs:1,0); %a=1, b=2
				\draw [line width=0.3mm] (axis cs:1,1) -- (axis cs:2,1);
				\draw [line width=0.3mm] (axis cs:2,0) -- (axis cs:5,0);
				\draw [line width=0.1mm, dashed] (axis cs:0,1) -- (axis cs:1,1);
				\draw [line width=0.1mm, dashed] (axis cs:1,1) -- (axis cs:1,0);
				\draw [line width=0.1mm, dashed] (axis cs:2,1) -- (axis cs:2,0);

				\addlegendentry{$a=1$, $b=2$}

				\addplot [only marks, mark=*] table {
				1 1
				2 1
				};

				\addplot [draw=none, forget plot] coordinates {(-1,-0.2)};
				\addplot [draw=none, forget plot] coordinates {(4,1.4)};

				\addplot [draw=none, forget plot] coordinates {(0,1)} node[left] {\small{$\frac{1}{b-a}=1$}};
				%\addplot [draw=none, forget plot] coordinates {(1,0)} node[below] {\small{$a$}};
				%\addplot [draw=none, forget plot] coordinates {(2,0)} node[below] {\small{$b$}};


				\draw [line width=0.3mm, black2] (axis cs:-2,0) -- (axis cs:-1,0); %a=-1, b=3
				\draw [line width=0.3mm, black2] (axis cs:-1,0.25) -- (axis cs:3,0.25);
				\draw [line width=0.3mm, black2] (axis cs:3,0) -- (axis cs:5,0);
				\draw [line width=0.1mm, black2, dashed] (axis cs:-1,0) -- (axis cs:-1,0.25);
				\draw [line width=0.1mm, black2, dashed] (axis cs:3,0) -- (axis cs:3,0.25);

				\addlegendentry{$a=-1$, $b=3$}

				\addplot [only marks, mark=*, black2] table {
				-1 0.25
				3 0.25
				};

				%\addplot [draw=none, forget plot] coordinates {(-1,-0.2)}; %togliere?
				%\addplot [draw=none, forget plot] coordinates {(4,1.4)}; %togliere?

				\addplot [draw=none, forget plot, black2] coordinates {(0,0.45)} node[left] {\small{$\frac{1}{b-a}=\frac{1}{4}$}};
				%\addplot [draw=none, forget plot, black2] coordinates {(-1,0)} node[below] {\small{$a$}};
				%\addplot [draw=none, forget plot, black2] coordinates {(3,0)} node[below] {\small{$b$}};

			\end{axis}

		\end{tikzpicture}
		\hskip 1pt
		\begin{tikzpicture}

			\begin{axis}[
			axis lines = middle,
			ylabel = $F(t)$,
			xlabel = $t$,
			width=0.57\textwidth,
			height=0.5\textwidth,
			xmin=-2,
			xmax= 5,
			ymin=-0.3,
			ymax= 1.5,
			yticklabels={,,},
			%xticklabels={,,}
			]

				\draw [line width=0.3mm] (axis cs:-2,0) -- (axis cs:1,0);
				\draw [line width=0.3mm] (axis cs:1,0) -- (axis cs:2,1);
				\draw [line width=0.3mm] (axis cs:2,1) -- (axis cs:5,1);
				\draw [line width=0.1mm, dashed] (axis cs:0,1) -- (axis cs:2,1);

%				\addlegendentry{$a=1$, $b=2$}

				\addplot [draw=none, forget plot] coordinates {(-1,-0.2)};
				\addplot [draw=none, forget plot] coordinates {(4,1.4)};

				\addplot [draw=none, forget plot] coordinates {(0,1)} node[left] {\small{$1$}};
				%\addplot [draw=none, forget plot] coordinates {(1,0)} node[below] {\small{$a$}};
				%\addplot [draw=none, forget plot] coordinates {(2,0)} node[below] {\small{$b$}};


				\draw [line width=0.3mm, black2] (axis cs:-2,0) -- (axis cs:-1,0);
				\draw [line width=0.3mm, black2] (axis cs:.-1,0) -- (axis cs:3,1);
				\draw [line width=0.3mm, black2] (axis cs:3,1) -- (axis cs:5,1);

%				\addlegendentry{$a=-1$, $b=3$}

				%\addplot [draw=none, forget plot, black2] coordinates {(-1,0)} node[below] {\small{$a$}};
				%\addplot [draw=none, forget plot, black2] coordinates {(3,0)} node[below] {\small{$b$}};

			\end{axis}

		\end{tikzpicture}

%		\caption{grafici della funzione di densità e della funzione di ripartizione per la distribuzione uniforme continua.}
	\end{figure}


	\begin{tabular*}{1\textwidth}{l l l}
		\textbf{Supporto:} & $[a,b] \in \RR$ & \\ \hline
		\textbf{Funzione di densità:}    &  $f(x) = \begin{cases} \dfrac{1}{b-a} & \text{per }x \in [a,b] \\ 0 & \text{altrimenti} \end{cases}$ \CS[0.8]{0.6}\\ \hline
		\textbf{Funzione di ripartizione:}    &  $F(t) = \begin{cases} 0 & \text{per } t < a \\ \dfrac{t-a}{b-a} & \text{per }t \in [a,b] \\ 1 & \text{per }t > b \end{cases}$ \CS[1]{0.8}\\ \hline
		\textbf{Funzione caratteristica:} & $ \varphi(u) = \exp\left\{i \dfrac{a+b}{2} u\right\} \dfrac{\sin\left(\dfrac{b-a}{2}u\right)}{\left(\dfrac{b-a}{2}u\right)}$ \CS[1.08]{0.88}\\ \hline
		\textbf{Valore atteso:} & $\EE[X] = \frac{1}{2}(a+b)$ &\CS[0.5]{0.3} \\ \hline
		\textbf{Varianza:} & $Var (X) = \frac{1}{12}(b-a)^2$ &\CS[0.5]{0.3} \\
	\end{tabular*}

\clearpage

%%%%%%%%%%%%%%%%%%%%%%%%%%%%%%%%%%%%%%%%%%%%%%%%%%

\needspace{7\baselineskip}
\subsection{Esponenziale}

	$$ \Ec (\lambda) $$

	Descrive il tempo di attesa al verificarsi di un successo che avviene periodicamente: misura dunque la durata tra eventi in un processo di Poisson di parametro $\lambda$. In quanto priva di memoria, può modellizzare il tempo di vita di oggetti che non invecchiano.

	\begin{figure}[H]

		\centering

		\begin{tikzpicture}

			\begin{axis}[
			axis lines = middle,
			ylabel = $f(x)$,
			xlabel = $x$,
			width=0.5\textwidth,
			height=0.5\textwidth,
			%yticklabels={,,},
			%xticklabels={,,}
			]

				\addplot [
				domain=0:5,
				samples=100,
				color=black,
				line width=0.3mm
				]
				{0.5*e^(-0.5*x)};

				%\addlegendentry{$\lambda = 1$}

				\addplot [
				domain=0:5,
				samples=100,
				color=black2,
				line width=0.3mm
				]
				{e^(-x)};

				%\addlegendentry{$\lambda = 2$}

				\addplot [
				domain=0:5,
				samples=100,
				color=black3,
				line width=0.3mm
				]
				{2*e^(-2*x)};

				%\addlegendentry{$\lambda = 5$}

				\addplot [draw=none, forget plot] coordinates {(1, 0.5)};

			\end{axis}

		\end{tikzpicture}
		\hskip 1pt
		\begin{tikzpicture}

			\begin{axis}[
			axis lines = middle,
			ylabel = $F(t)$,
			xlabel = $t$,
			width=0.5\textwidth,
			height=0.5\textwidth,
			%yticklabels={,,},
			%xticklabels={,,}
			legend style={at={(axis cs: 5,0.1)},anchor=south east, font=\tiny},
			legend cell align=left,
			]

				\addplot [
				domain=0:5,
				samples=100,
				color=black,
				line width=0.3mm
				]
				{1-e^(-0.5*x)};

				\addlegendentry{$\lambda = 1$}

				\addplot [
				domain=0:5,
				samples=100,
				color=black2,
				line width=0.3mm
				]
				{1-e^(-1*x)};

				\addlegendentry{$\lambda = 2$}

				\addplot [
				domain=0:5,
				samples=100,
				color=black3,
				line width=0.3mm
				]
				{1-e^(-2*x)};

				\addlegendentry{$\lambda = 5$}

				\addplot [draw=none, forget plot] coordinates {(1, 0.5)};

			\end{axis}

		\end{tikzpicture}

%		\caption{grafici della funzione di densità e della funzione di ripartizione per la distribuzione esponenziale.}

	\end{figure}

	\begin{tabular*}{1\textwidth}{l l l}
		\textbf{Supporto:} & $[0,+\infty)$ & \CS{0.40}\\ \hline
		\textbf{Funzione di densità:}    &  $f(x) = \lambda e^{-\lambda x} \Ind_{(0, +\infty)}(x)$\CS[0.60]{0.40}\\ \hline
		\textbf{Funzione di ripartizione:}    &  $F(t) = 1- e^{-\lambda t} \Ind_{(0, +\infty)}(t)$\CS[0.60]{0.40}\\ \hline
		\textbf{Funzione caratteristica:} & $\varphi(u)=\dfrac{\lambda}{\lambda-iu}$ & \CS[0.60]{0.40}\\ \hline
		\textbf{Valore atteso:} & $\EE[X]= \dfrac{1}{\lambda}$ & \CS[0.60]{0.40}\\ \hline
		\textbf{Varianza:} & $Var(X)= \dfrac{1}{\lambda^2}$ & \CS[0.60]{0.40}\\
	\end{tabular*}

\clearpage

%%%%%%%%%%%%%%%%%%%%%%%%%%%%%%%%%%%%%%%%%%%%%%%%%%

\needspace{7\baselineskip}
\subsection{Normale}

	$$ \Nc(\mu, \sigma^2) $$

	Detta anche \textbf{distribuzione gaussiana}, riveste un ruolo cruciale nella teoria della probabilità e della statistica.
	I suoi parametri $\mu$ e $\sigma^2$ coincidono rispettivamente con la sua media e varianza.
%	Il teorema centrale del limite (\ref{TCL}) afferma che una somma di $n$ VA con media e varianza finite converge in legge, al tendere di $n$ all'infinito, a una distribuzione normale.
%	\clearpage
%	A causa di ciò questa distribuzione dalla tipica forma a campana rappresenta un modello largamente impiegato, anche per rappresentare fenomeni complessi.
	Nel caso di media nulla e varianza unitaria si ha una \textbf{normale standard}: $Z \sim \Nc(0,1)$.

	\begin{figure}[H]

		\centering

		\begin{tikzpicture}

			\begin{axis}[
			axis lines = middle,
			ylabel = $f(x)$,
			xlabel = $x$,
			width=0.6\textwidth,
			height=0.6\textwidth,
			%yticklabels={,,},
			%xticklabels={,,},
			%axis equal=true
			]

				\addplot [
				domain=-5:5,
				samples=100,
				color=black,
				line width=0.3mm
				] {
					1/sqrt(2*pi)*e^(-(x)^2/2
					};

				%\addlegendentry{$\mu =0, \sigma^2=0$}

				\addplot [
				domain=-5:5,
				samples=100,
				color=black2,
				line width=0.3mm
				] {
					1/sqrt(2*pi*2)*e^(-(x-1)^2/4
					};

				%\addlegendentry{$\mu =1, \sigma^2=2$}

				\addplot [
				domain=-5:5,
				samples=100,
				color=black3,
				line width=0.3mm
				] {
					1/sqrt(2*pi*5)*e^(-(x+1)^2/10
					};

				%\addlegendentry{$\mu =-1, \sigma^2=5$}

				\addplot [draw=none, forget plot] coordinates {(1, 0.5)};

			\end{axis}

		\end{tikzpicture}
		\hskip 1pt
		\begin{tikzpicture}[declare function={erf(\x)=%
		(1+(e^(-(\x*\x))*(-265.057+abs(\x)*(-135.065+abs(\x)%
		*(-59.646+(-6.84727-0.777889*abs(\x))*abs(\x)))))%
			/(3.05259+abs(\x))^5)*(\x>0?1:-1);}]

			\begin{axis}[
			%axis lines = middle,
			axis x line= center,
			axis y line= center,
			ylabel = $F(t)$,
			xlabel = $t$,
			width=0.6\textwidth,
			height=0.6\textwidth,
			%yticklabels={,,},
			%xticklabels={,,},
			%axis equal=true		%Il Signore mi perdoni
			legend style={at={(axis cs: 5,0.07)},anchor=south east, font=\tiny},
			legend cell align=left,
			]

				\addplot [
				domain=-5:5,
				samples=100,
				color=black,
				line width=0.3mm
				] {
					(1/2)*(1+erf((x-0)/(1*sqrt(2))))
					};

				\addlegendentry{$\mu =0, \sigma^2=0$}

				\addplot [
				domain=-5:5,
				samples=100,
				color=black2,
				line width=0.3mm
				] {
					(1/2)*(1+erf((x-1)/(1*sqrt(2*2))))
					};

				\addlegendentry{$\mu =1, \sigma^2=2$}

				\addplot [
				domain=-5:5,
				samples=100,
				color=black3,
				line width=0.3mm
				] {
					(1/2)*(1+erf((x+1)/(1*sqrt(2*5))))
					};

				\addlegendentry{$\mu =-1, \sigma^2=5$}

				%\addplot [draw=none, forget plot] coordinates {(1, 0.5)};

			\end{axis}

		\end{tikzpicture}

%		\caption{grafici della funzione di densità e della funzione di ripartizione per la distribuzione normale.}

	\end{figure}

	\begin{tabular*}{1\textwidth}{l l l}
		\textbf{Supporto:} & $\RR$ & \\ \hline
		\textbf{Funzione di densità:} & $f(x)= \dfrac{1}{\sqrt{2 \pi \sigma^2}}\exp\left\{- \dfrac{(x-\mu)^2}{2 \sigma^2}\right\}$ & \CS[0.6]{0.4} \\ \hline
		\textbf{Funzione di ripartizione:} & $F(t) = \Phi \left(\dfrac{t-\mu}{\sigma} \right) = \frac{1}{2}\left[1+\operatorname{erf}\left( \frac{t-\mu}{\sqrt{2}\sigma} \right)\right] $  & \CS[0.58]{0.4}\\ \hline
		\textbf{Funzione caratteristica:} & $\varphi(u) = \exp\left\{i u \mu-\frac{\sigma^2}{2}u^2\right\}$ & \CS[0.5]{0.3}\\ \hline
		\textbf{Valore atteso:} & $\EE[X]=\mu$ & \\ \hline
		\textbf{Varianza:} & $Var(X)= \sigma^2$ & \\ \hline
		\textbf{Momento quarto:} & $\EE[(X-\mu)^4] = 3\sigma^4$ & \\ \hline
%		\textbf{Trasformazioni affini:} & {Date $X \sim N(\mu_X, \sigma^2_X)$, $Y \sim N(\mu_Y, \sigma^2_Y)$}\\
%		& {$AX+b \sim \Nc (A \mu_X + b, A\sigma^2_XA^T)$ }&\\
%		& {$aX+bY \sim N(a\mu_X + b\mu_Y, a^2\sigma^2_X + b^2\sigma^2_Y + 2ab\sigma_{XY})$}\\
%		& {$a^2\sigma^2_X + b^2\sigma^2_Y + 2ab\sigma_{XY} \geq 0$ (Condizione da verificare)}
		\textbf{Trasformazioni affini:} & \makecell[l] {\scriptsize{Date $X \sim N(\mu_X, \sigma^2_X)$, $Y \sim N(\mu_Y, \sigma^2_Y)$}\\
		\scriptsize{$AX+b \sim \Nc (A \mu_X + b, A\sigma^2_XA^T)$ }\\
		\scriptsize{$aX+bY \sim N(a\mu_X + b\mu_Y, a^2\sigma^2_X + b^2\sigma^2_Y + 2ab\sigma_{XY})$}\\
		\scriptsize{$a^2\sigma^2_X + b^2\sigma^2_Y + 2ab\sigma_{XY} \geq 0$ (Condizione da verificare)}}
	\end{tabular*}


\clearpage
%%%%%%%%%%%%%%%%%%%%%%%%%%%%%%%%%%%%%%%%%%%%%%%%%%

\needspace{7\baselineskip}
\subsection{Gamma}

	$$ \Gamma(\alpha, \beta) $$

	Comprende come casi particolari le distribuzioni esponenziali e chi-quadro. È definita come la somma di VA indipendenti con distribuzione esponenziale ($\alpha$ VA esponenziali di parametro $\beta$), con $\alpha>0$ e $\beta>0$.

	\begin{figure}[H]

		\centering

		\begin{tikzpicture}

	\begin{axis}[
	axis lines = middle,
	ylabel = $f(x)$,
	xlabel = $x$,
	width=0.5\textwidth,
	height=0.5\textwidth,
	%yticklabels={,,},
	%xticklabels={,,},
	%legend style={at={(axis cs: 10,0.35)},anchor=south east, font=\tiny},
	]
	
		\addplot [
		smooth,
		domain=0:10,
		samples=100,
		color=black,
		line width=0.3mm
		] coordinates {
			(	0	,	0.5000	)
			(	0.2041	,	0.4515	)
			(	0.4082	,	0.4077	)
			(	0.6122	,	0.3681	)
			(	0.8163	,	0.3324	)
			(	1.0204	,	0.3002	)
			(	1.2245	,	0.2711	)
			(	1.4286	,	0.2448	)
			(	1.6327	,	0.2210	)
			(	1.8367	,	0.1996	)
			(	2.0408	,	0.1802	)
			(	2.2449	,	0.1627	)
			(	2.4490	,	0.1470	)
			(	2.6531	,	0.1327	)
			(	2.8571	,	0.1198	)
			(	3.0612	,	0.1082	)
			(	3.2653	,	0.0977	)
			(	3.4694	,	0.0882	)
			(	3.6735	,	0.0797	)
			(	3.8776	,	0.0719	)
			(	4.0816	,	0.0650	)
			(	4.2857	,	0.0587	)
			(	4.4898	,	0.0530	)
			(	4.6939	,	0.0478	)
			(	4.8980	,	0.0432	)
			(	5.1020	,	0.0390	)
			(	5.3061	,	0.0352	)
			(	5.5102	,	0.0318	)
			(	5.7143	,	0.0287	)
			(	5.9184	,	0.0259	)
			(	6.1224	,	0.0234	)
			(	6.3265	,	0.0211	)
			(	6.5306	,	0.0191	)
			(	6.7347	,	0.0172	)
			(	6.9388	,	0.0156	)
			(	7.1429	,	0.0141	)
			(	7.3469	,	0.0127	)
			(	7.5510	,	0.0115	)
			(	7.7551	,	0.0104	)
			(	7.9592	,	0.0093	)
			(	8.1633	,	0.0084	)
			(	8.3673	,	0.0076	)
			(	8.5714	,	0.0069	)
			(	8.7755	,	0.0062	)
			(	8.9796	,	0.0056	)
			(	9.1837	,	0.0051	)
			(	9.3878	,	0.0046	)
			(	9.5918	,	0.0041	)
			(	9.7959	,	0.0037	)
			(	10.0000	,	0.0034	)
			};
		
		%\addlegendentry{$\alpha =1, \, \beta=0.5$}
		
		\addplot [
		smooth,
		domain=0:10,
		samples=100,
		color=black2,
		line width=0.3mm
		] coordinates {
			(	0	,	0	)
			(	0.2041	,	0.0012	)
			(	0.4082	,	0.0075	)
			(	0.6122	,	0.0207	)
			(	0.8163	,	0.0401	)
			(	1.0204	,	0.0638	)
			(	1.2245	,	0.0899	)
			(	1.4286	,	0.1164	)
			(	1.6327	,	0.1417	)
			(	1.8367	,	0.1646	)
			(	2.0408	,	0.1841	)
			(	2.2449	,	0.1998	)
			(	2.4490	,	0.2115	)
			(	2.6531	,	0.2192	)
			(	2.8571	,	0.2233	)
			(	3.0612	,	0.2239	)
			(	3.2653	,	0.2216	)
			(	3.4694	,	0.2167	)
			(	3.6735	,	0.2098	)
			(	3.8776	,	0.2012	)
			(	4.0816	,	0.1913	)
			(	4.2857	,	0.1806	)
			(	4.4898	,	0.1693	)
			(	4.6939	,	0.1577	)
			(	4.8980	,	0.1461	)
			(	5.1020	,	0.1347	)
			(	5.3061	,	0.1235	)
			(	5.5102	,	0.1128	)
			(	5.7143	,	0.1026	)
			(	5.9184	,	0.0929	)
			(	6.1224	,	0.0839	)
			(	6.3265	,	0.0755	)
			(	6.5306	,	0.0677	)
			(	6.7347	,	0.0605	)
			(	6.9388	,	0.0540	)
			(	7.1429	,	0.0480	)
			(	7.3469	,	0.0426	)
			(	7.5510	,	0.0377	)
			(	7.7551	,	0.0333	)
			(	7.9592	,	0.0294	)
			(	8.1633	,	0.0258	)
			(	8.3673	,	0.0227	)
			(	8.5714	,	0.0199	)
			(	8.7755	,	0.0174	)
			(	8.9796	,	0.0152	)
			(	9.1837	,	0.0133	)
			(	9.3878	,	0.0115	)
			(	9.5918	,	0.0100	)
			(	9.7959	,	0.0087	)
			(	10.0000	,	0.0076	)
		};
		
		%\addlegendentry{$\alpha =4, \, \beta=1$}
		
		\addplot [
		smooth,
		domain=0:10,
		samples=100,
		color=black3,
		line width=0.3mm
		] coordinates {
			(	0	,	0	)
			(	0.2041	,	0.0000	)
			(	0.4082	,	0.0000	)
			(	0.6122	,	0.0001	)
			(	0.8163	,	0.0005	)
			(	1.0204	,	0.0019	)
			(	1.2245	,	0.0055	)
			(	1.4286	,	0.0127	)
			(	1.6327	,	0.0245	)
			(	1.8367	,	0.0418	)
			(	2.0408	,	0.0645	)
			(	2.2449	,	0.0919	)
			(	2.4490	,	0.1226	)
			(	2.6531	,	0.1546	)
			(	2.8571	,	0.1860	)
			(	3.0612	,	0.2148	)
			(	3.2653	,	0.2393	)
			(	3.4694	,	0.2584	)
			(	3.6735	,	0.2714	)
			(	3.8776	,	0.2781	)
			(	4.0816	,	0.2787	)
			(	4.2857	,	0.2738	)
			(	4.4898	,	0.2641	)
			(	4.6939	,	0.2506	)
			(	4.8980	,	0.2342	)
			(	5.1020	,	0.2158	)
			(	5.3061	,	0.1964	)
			(	5.5102	,	0.1766	)
			(	5.7143	,	0.1571	)
			(	5.9184	,	0.1383	)
			(	6.1224	,	0.1206	)
			(	6.3265	,	0.1042	)
			(	6.5306	,	0.0893	)
			(	6.7347	,	0.0760	)
			(	6.9388	,	0.0641	)
			(	7.1429	,	0.0538	)
			(	7.3469	,	0.0448	)
			(	7.5510	,	0.0371	)
			(	7.7551	,	0.0305	)
			(	7.9592	,	0.0250	)
			(	8.1633	,	0.0203	)
			(	8.3673	,	0.0165	)
			(	8.5714	,	0.0133	)
			(	8.7755	,	0.0107	)
			(	8.9796	,	0.0085	)
			(	9.1837	,	0.0068	)
			(	9.3878	,	0.0054	)
			(	9.5918	,	0.0042	)
			(	9.7959	,	0.0033	)
			(	10.0000	,	0.0026	)
		};
		
		%\addlegendentry{$\alpha =9, \, \beta=2$}
		
		\addplot [draw=none, forget plot] coordinates {(1, 0.5)};
	
	\end{axis}
	
\end{tikzpicture}
		\hskip 1pt
		\begin{tikzpicture}

	\begin{axis}[
	axis lines = middle,
	ylabel = $F(t)$,
	xlabel = $t$,
	width=0.5\textwidth,
	height=0.5\textwidth,
	%yticklabels={,,},
	%xticklabels={,,},
	legend style={at={(axis cs: 10,0.1)},anchor=south east, font=\tiny},
	legend cell align=left,
	]
	
		\addplot [
		smooth,
		domain=0:10,
		samples=100,
		color=black,
		line width=0.3mm
		] coordinates 	{
			(	0	,	0	)
			(	0.2041	,	0.0970	)
			(	0.4082	,	0.1846	)
			(	0.6122	,	0.2637	)
			(	0.8163	,	0.3351	)
			(	1.0204	,	0.3996	)
			(	1.2245	,	0.4579	)
			(	1.4286	,	0.5105	)
			(	1.6327	,	0.5579	)
			(	1.8367	,	0.6008	)
			(	2.0408	,	0.6396	)
			(	2.2449	,	0.6745	)
			(	2.4490	,	0.7061	)
			(	2.6531	,	0.7346	)
			(	2.8571	,	0.7603	)
			(	3.0612	,	0.7836	)
			(	3.2653	,	0.8046	)
			(	3.4694	,	0.8235	)
			(	3.6735	,	0.8407	)
			(	3.8776	,	0.8561	)
			(	4.0816	,	0.8701	)
			(	4.2857	,	0.8827	)
			(	4.4898	,	0.8941	)
			(	4.6939	,	0.9043	)
			(	4.8980	,	0.9136	)
			(	5.1020	,	0.9220	)
			(	5.3061	,	0.9296	)
			(	5.5102	,	0.9364	)
			(	5.7143	,	0.9426	)
			(	5.9184	,	0.9481	)
			(	6.1224	,	0.9532	)
			(	6.3265	,	0.9577	)
			(	6.5306	,	0.9618	)
			(	6.7347	,	0.9655	)
			(	6.9388	,	0.9689	)
			(	7.1429	,	0.9719	)
			(	7.3469	,	0.9746	)
			(	7.5510	,	0.9771	)
			(	7.7551	,	0.9793	)
			(	7.9592	,	0.9813	)
			(	8.1633	,	0.9831	)
			(	8.3673	,	0.9848	)
			(	8.5714	,	0.9862	)
			(	8.7755	,	0.9876	)
			(	8.9796	,	0.9888	)
			(	9.1837	,	0.9899	)
			(	9.3878	,	0.9908	)
			(	9.5918	,	0.9917	)
			(	9.7959	,	0.9925	)
			(	10.0000	,	0.9933	)
		};
	
		\addlegendentry{$\alpha =1, \, \beta=0.5$}
		
		\addplot [
		smooth,
		domain=0:10,
		samples=100,
		color=black2,
		line width=0.3mm
		] coordinates 	{
			(	0	,	0	)
			(	0.2041	,	0.0001	)
			(	0.4082	,	0.0008	)
			(	0.6122	,	0.0036	)
			(	0.8163	,	0.0097	)
			(	1.0204	,	0.0203	)
			(	1.2245	,	0.0359	)
			(	1.4286	,	0.0570	)
			(	1.6327	,	0.0834	)
			(	1.8367	,	0.1147	)
			(	2.0408	,	0.1503	)
			(	2.2449	,	0.1895	)
			(	2.4490	,	0.2316	)
			(	2.6531	,	0.2756	)
			(	2.8571	,	0.3208	)
			(	3.0612	,	0.3665	)
			(	3.2653	,	0.4120	)
			(	3.4694	,	0.4567	)
			(	3.6735	,	0.5003	)
			(	3.8776	,	0.5422	)
			(	4.0816	,	0.5823	)
			(	4.2857	,	0.6203	)
			(	4.4898	,	0.6560	)
			(	4.6939	,	0.6894	)
			(	4.8980	,	0.7204	)
			(	5.1020	,	0.7490	)
			(	5.3061	,	0.7753	)
			(	5.5102	,	0.7995	)
			(	5.7143	,	0.8214	)
			(	5.9184	,	0.8414	)
			(	6.1224	,	0.8594	)
			(	6.3265	,	0.8756	)
			(	6.5306	,	0.8902	)
			(	6.7347	,	0.9033	)
			(	6.9388	,	0.9150	)
			(	7.1429	,	0.9254	)
			(	7.3469	,	0.9346	)
			(	7.5510	,	0.9428	)
			(	7.7551	,	0.9500	)
			(	7.9592	,	0.9564	)
			(	8.1633	,	0.9621	)
			(	8.3673	,	0.9670	)
			(	8.5714	,	0.9713	)
			(	8.7755	,	0.9751	)
			(	8.9796	,	0.9785	)
			(	9.1837	,	0.9814	)
			(	9.3878	,	0.9839	)
			(	9.5918	,	0.9861	)
			(	9.7959	,	0.9880	)
			(	10.0000	,	0.9897	)
		};
		
		\addlegendentry{$\alpha =4, \, \beta=1$}
		
		\addplot [
		smooth,
		domain=0:10,
		samples=100,
		color=black3,
		line width=0.3mm
		] coordinates 	{
			(	0	,	0	)
			(	0.2041	,	0.0000	)
			(	0.4082	,	0.0000	)
			(	0.6122	,	0.0000	)
			(	0.8163	,	0.0001	)
			(	1.0204	,	0.0003	)
			(	1.2245	,	0.0010	)
			(	1.4286	,	0.0028	)
			(	1.6327	,	0.0065	)
			(	1.8367	,	0.0131	)
			(	2.0408	,	0.0239	)
			(	2.2449	,	0.0398	)
			(	2.4490	,	0.0616	)
			(	2.6531	,	0.0899	)
			(	2.8571	,	0.1247	)
			(	3.0612	,	0.1657	)
			(	3.2653	,	0.2121	)
			(	3.4694	,	0.2630	)
			(	3.6735	,	0.3171	)
			(	3.8776	,	0.3733	)
			(	4.0816	,	0.4302	)
			(	4.2857	,	0.4867	)
			(	4.4898	,	0.5417	)
			(	4.6939	,	0.5942	)
			(	4.8980	,	0.6437	)
			(	5.1020	,	0.6897	)
			(	5.3061	,	0.7318	)
			(	5.5102	,	0.7698	)
			(	5.7143	,	0.8039	)
			(	5.9184	,	0.8340	)
			(	6.1224	,	0.8604	)
			(	6.3265	,	0.8833	)
			(	6.5306	,	0.9030	)
			(	6.7347	,	0.9198	)
			(	6.9388	,	0.9341	)
			(	7.1429	,	0.9461	)
			(	7.3469	,	0.9562	)
			(	7.5510	,	0.9645	)
			(	7.7551	,	0.9714	)
			(	7.9592	,	0.9770	)
			(	8.1633	,	0.9816	)
			(	8.3673	,	0.9854	)
			(	8.5714	,	0.9884	)
			(	8.7755	,	0.9908	)
			(	8.9796	,	0.9928	)
			(	9.1837	,	0.9943	)
			(	9.3878	,	0.9956	)
			(	9.5918	,	0.9965	)
			(	9.7959	,	0.9973	)
			(	10.0000	,	0.9979	)
		};
		
		\addlegendentry{$\alpha =9, \, \beta=2$}
		
		%\addplot [draw=none, forget plot] coordinates {(1, 0.5)};
	
	\end{axis}
	
\end{tikzpicture}

%		\caption{grafici della funzione di densità e della funzione di ripartizione per la distribuzione gamma.}

	\end{figure}

	\begin{tabular*}{1\textwidth}{l l l}
		\textbf{Supporto:} & $[0,+\infty)$ &\CS{0.40} \\ \hline
		\textbf{Funzione di densità:} & $f(x)= \dfrac{\beta^\alpha x^{\alpha-1} e^{-\beta x}}{\Gamma(\alpha)}\Ind_{(0, +\infty)}(x)$& \CS[0.62]{0.45}\\ \hline
		\textbf{Funzione di ripartizione:}    &  $F(t) = \dfrac{\gamma(\alpha, \beta t)}{\Gamma(\alpha)} $ & \CS[0.62]{0.42}\\ \hline
		%\textbf{Funzione di ripartizione:}    & $F(t)= \left(1- \sum\limits_{i=0}^{\alpha-1} e^{- \beta t} \dfrac{(\beta t)^i}{i!} \right)$ se $\alpha, \; \beta \in \NN$ & \CS[0.62]{0.42}\\ \hline
		\textbf{Funzione caratteristica:} & $\varphi(u)= \left( \dfrac{ \beta }{\beta - iu} \right)^\alpha$ &\CS[0.62]{0.42}\\ \hline
		\textbf{Valore atteso e momenti:} & $\EE [X^k] = \dfrac{\alpha(\alpha+1) \ldots (\alpha + k -1)}{\beta^k} $ & \CS[0.6]{0.40}\\ \hline
		\textbf{Varianza:} & $Var(X)= \frac{\alpha}{\beta^2}$ & \CS[0.55]{0.35} \\ \hline
		\textbf{Prodotto per scalare:} & $X \sim \Gamma(\alpha, \beta), \ Y= cX, \ c>0 \implies Y \sim \Gamma(\alpha, \frac{\beta}{c})$& \CS[0.55]{0.35} \\
	\end{tabular*}
	Dove la \textit{funzione} Gamma di Eulero è definita come:
	$$ \Gamma(x) = \int_0^{+\infty} t^{x-1}e^{-t}\de t $$
	
\clearpage

	Un caso particolare della distribuzione gamma è la distribuzione erlanghiana (da A. K.  Erlang) dove i parametri $\alpha$ e $\beta$ sono interi positivi. Solo in questo caso la funzione di ripartizione si può scrivere come: $$F(t)= \left(1- \sum\limits_{i=0}^{\alpha-1} e^{- \beta t} \dfrac{(\beta t)^i}{i!} \right)$$

\clearpage
%%%%%%%%%%%%%%%%%%%%%%%%%%%%%%%%%%%%%%%%%%%%%%%%%%

\needspace{7\baselineskip}
\subsection{Chi-quadro}

	$$ \chi^2 (k) = \Gamma \left( \frac{k}{2}, \frac{1}{2} \right) = \sum_{i=1}^{k} Z_i^2$$

	Somma di $k$ normali standard, ovvero $Z \sim \Nc(0,1)$, ciascuna elevata al quadrato.

	\begin{figure}[H]

		\centering

		\begin{tikzpicture}

	\begin{axis}[
	axis lines = middle,
	ylabel = $f(x)$,
	xlabel = $x$,
	width=0.5\textwidth,
	height=0.5\textwidth,
	%yticklabels={,,},
	%xticklabels={,,}
	]
	
		\addplot [ 
		smooth,
		domain=0:10,
		samples=100,
		color=black,
		line width=0.3mm
		] coordinates {
			%(	0	,	Inf	)hahaha che burlone
			(	0.2041	,	0.7974	)
			(	0.4082	,	0.5092	)
			(	0.6122	,	0.3754	)
			(	0.8163	,	0.2936	)
			(	1.0204	,	0.2371	)
			(	1.2245	,	0.1955	)
			(	1.4286	,	0.1634	)
			(	1.6327	,	0.1380	)
			(	1.8367	,	0.1175	)
			(	2.0408	,	0.1007	)
			(	2.2449	,	0.0867	)
			(	2.4490	,	0.0749	)
			(	2.6531	,	0.0650	)
			(	2.8571	,	0.0566	)
			(	3.0612	,	0.0493	)
			(	3.2653	,	0.0431	)
			(	3.4694	,	0.0378	)
			(	3.6735	,	0.0332	)
			(	3.8776	,	0.0291	)
			(	4.0816	,	0.0257	)
			(	4.2857	,	0.0226	)
			(	4.4898	,	0.0199	)
			(	4.6939	,	0.0176	)
			(	4.8980	,	0.0156	)
			(	5.1020	,	0.0138	)
			(	5.3061	,	0.0122	)
			(	5.5102	,	0.0108	)
			(	5.7143	,	0.0096	)
			(	5.9184	,	0.0085	)
			(	6.1224	,	0.0076	)
			(	6.3265	,	0.0067	)
			(	6.5306	,	0.0060	)
			(	6.7347	,	0.0053	)
			(	6.9388	,	0.0047	)
			(	7.1429	,	0.0042	)
			(	7.3469	,	0.0037	)
			(	7.5510	,	0.0033	)
			(	7.7551	,	0.0030	)
			(	7.9592	,	0.0026	)
			(	8.1633	,	0.0024	)
			(	8.3673	,	0.0021	)
			(	8.5714	,	0.0019	)
			(	8.7755	,	0.0017	)
			(	8.9796	,	0.0015	)
			(	9.1837	,	0.0013	)
			(	9.3878	,	0.0012	)
			(	9.5918	,	0.0011	)
			(	9.7959	,	0.0010	)
			(	10.0000	,	0.0009	)
			};
		
%		\addlegendentry{$k=1$}
		
		\addplot [ 
		smooth,
		domain=0:10,
		samples=100,
		color=black2,
		line width=0.3mm
		] coordinates {
			(	0	,	0	)
			(	0.2041	,	0.1627	)
			(	0.4082	,	0.2078	)
			(	0.6122	,	0.2298	)
			(	0.8163	,	0.2397	)
			(	1.0204	,	0.2419	)
			(	1.2245	,	0.2393	)
			(	1.4286	,	0.2334	)
			(	1.6327	,	0.2253	)
			(	1.8367	,	0.2158	)
			(	2.0408	,	0.2054	)
			(	2.2449	,	0.1946	)
			(	2.4490	,	0.1835	)
			(	2.6531	,	0.1725	)
			(	2.8571	,	0.1616	)
			(	3.0612	,	0.1511	)
			(	3.2653	,	0.1409	)
			(	3.4694	,	0.1311	)
			(	3.6735	,	0.1218	)
			(	3.8776	,	0.1130	)
			(	4.0816	,	0.1047	)
			(	4.2857	,	0.0969	)
			(	4.4898	,	0.0896	)
			(	4.6939	,	0.0827	)
			(	4.8980	,	0.0763	)
			(	5.1020	,	0.0703	)
			(	5.3061	,	0.0647	)
			(	5.5102	,	0.0596	)
			(	5.7143	,	0.0548	)
			(	5.9184	,	0.0503	)
			(	6.1224	,	0.0462	)
			(	6.3265	,	0.0424	)
			(	6.5306	,	0.0389	)
			(	6.7347	,	0.0357	)
			(	6.9388	,	0.0327	)
			(	7.1429	,	0.0300	)
			(	7.3469	,	0.0275	)
			(	7.5510	,	0.0251	)
			(	7.7551	,	0.0230	)
			(	7.9592	,	0.0210	)
			(	8.1633	,	0.0192	)
			(	8.3673	,	0.0176	)
			(	8.5714	,	0.0161	)
			(	8.7755	,	0.0147	)
			(	8.9796	,	0.0134	)
			(	9.1837	,	0.0123	)
			(	9.3878	,	0.0112	)
			(	9.5918	,	0.0102	)
			(	9.7959	,	0.0093	)
			(	10.0000	,	0.0085	)
		};
		
%		\addlegendentry{$k=3$}
		
		\addplot [
		smooth,
		domain=0:10,
		samples=100,
		color=black3,
		line width=0.3mm
		] coordinates {
			(	0	,	0	)
			(	0.2041	,	0.0111	)
			(	0.4082	,	0.0283	)
			(	0.6122	,	0.0469	)
			(	0.8163	,	0.0652	)
			(	1.0204	,	0.0823	)
			(	1.2245	,	0.0977	)
			(	1.4286	,	0.1112	)
			(	1.6327	,	0.1226	)
			(	1.8367	,	0.1321	)
			(	2.0408	,	0.1397	)
			(	2.2449	,	0.1456	)
			(	2.4490	,	0.1498	)
			(	2.6531	,	0.1525	)
			(	2.8571	,	0.1539	)
			(	3.0612	,	0.1541	)
			(	3.2653	,	0.1533	)
			(	3.4694	,	0.1516	)
			(	3.6735	,	0.1492	)
			(	3.8776	,	0.1461	)
			(	4.0816	,	0.1425	)
			(	4.2857	,	0.1384	)
			(	4.4898	,	0.1340	)
			(	4.6939	,	0.1294	)
			(	4.8980	,	0.1245	)
			(	5.1020	,	0.1195	)
			(	5.3061	,	0.1145	)
			(	5.5102	,	0.1094	)
			(	5.7143	,	0.1043	)
			(	5.9184	,	0.0993	)
			(	6.1224	,	0.0943	)
			(	6.3265	,	0.0895	)
			(	6.5306	,	0.0847	)
			(	6.7347	,	0.0801	)
			(	6.9388	,	0.0757	)
			(	7.1429	,	0.0714	)
			(	7.3469	,	0.0672	)
			(	7.5510	,	0.0633	)
			(	7.7551	,	0.0595	)
			(	7.9592	,	0.0558	)
			(	8.1633	,	0.0524	)
			(	8.3673	,	0.0491	)
			(	8.5714	,	0.0459	)
			(	8.7755	,	0.0430	)
			(	8.9796	,	0.0402	)
			(	9.1837	,	0.0375	)
			(	9.3878	,	0.0350	)
			(	9.5918	,	0.0326	)
			(	9.7959	,	0.0304	)
			(	10.0000	,	0.0283	)
		};
		
%		\addlegendentry{$k=5$}
		
		\addplot [draw=none, forget plot] coordinates {(1, 0.5)};
	
	\end{axis}
	
\end{tikzpicture}
		\hskip 1pt
		\begin{tikzpicture}

	\begin{axis}[
	axis lines = middle,
	ylabel = $F(t)$,
	xlabel = $t$,
	width=0.5\textwidth,
	height=0.5\textwidth,
	%yticklabels={,,},
	%xticklabels={,,},
	legend style={at={(axis cs: 10,0.1)},anchor=south east, font=\tiny},
	legend cell align=left,
	]
	
		\addplot [ 
		smooth,
		domain=0:10,
		samples=100,
		color=black,
		line width=0.3mm
		] coordinates {
			(	0	,	0	)
			(	0.2041	,	0.3486	)
			(	0.4082	,	0.4771	)
			(	0.6122	,	0.5661	)
			(	0.8163	,	0.6337	)
			(	1.0204	,	0.6876	)
			(	1.2245	,	0.7315	)
			(	1.4286	,	0.7680	)
			(	1.6327	,	0.7987	)
			(	1.8367	,	0.8247	)
			(	2.0408	,	0.8469	)
			(	2.2449	,	0.8659	)
			(	2.4490	,	0.8824	)
			(	2.6531	,	0.8966	)
			(	2.8571	,	0.9090	)
			(	3.0612	,	0.9198	)
			(	3.2653	,	0.9292	)
			(	3.4694	,	0.9375	)
			(	3.6735	,	0.9447	)
			(	3.8776	,	0.9511	)
			(	4.0816	,	0.9566	)
			(	4.2857	,	0.9616	)
			(	4.4898	,	0.9659	)
			(	4.6939	,	0.9697	)
			(	4.8980	,	0.9731	)
			(	5.1020	,	0.9761	)
			(	5.3061	,	0.9787	)
			(	5.5102	,	0.9811	)
			(	5.7143	,	0.9832	)
			(	5.9184	,	0.9850	)
			(	6.1224	,	0.9867	)
			(	6.3265	,	0.9881	)
			(	6.5306	,	0.9894	)
			(	6.7347	,	0.9905	)
			(	6.9388	,	0.9916	)
			(	7.1429	,	0.9925	)
			(	7.3469	,	0.9933	)
			(	7.5510	,	0.9940	)
			(	7.7551	,	0.9946	)
			(	7.9592	,	0.9952	)
			(	8.1633	,	0.9957	)
			(	8.3673	,	0.9962	)
			(	8.5714	,	0.9966	)
			(	8.7755	,	0.9969	)
			(	8.9796	,	0.9973	)
			(	9.1837	,	0.9976	)
			(	9.3878	,	0.9978	)
			(	9.5918	,	0.9980	)
			(	9.7959	,	0.9983	)
			(	10.0000	,	0.9984	)
			};
		
		\addlegendentry{$k=1$}
		
		\addplot [ 
		smooth,
		domain=0:10,
		samples=100,
		color=black2,
		line width=0.3mm
		] coordinates {
			(	0	,	0	)
			(	0.2041	,	0.0231	)
			(	0.4082	,	0.0614	)
			(	0.6122	,	0.1064	)
			(	0.8163	,	0.1544	)
			(	1.0204	,	0.2037	)
			(	1.2245	,	0.2529	)
			(	1.4286	,	0.3011	)
			(	1.6327	,	0.3480	)
			(	1.8367	,	0.3930	)
			(	2.0408	,	0.4360	)
			(	2.2449	,	0.4768	)
			(	2.4490	,	0.5154	)
			(	2.6531	,	0.5517	)
			(	2.8571	,	0.5858	)
			(	3.0612	,	0.6177	)
			(	3.2653	,	0.6475	)
			(	3.4694	,	0.6752	)
			(	3.6735	,	0.7010	)
			(	3.8776	,	0.7250	)
			(	4.0816	,	0.7472	)
			(	4.2857	,	0.7678	)
			(	4.4898	,	0.7868	)
			(	4.6939	,	0.8044	)
			(	4.8980	,	0.8206	)
			(	5.1020	,	0.8355	)
			(	5.3061	,	0.8493	)
			(	5.5102	,	0.8620	)
			(	5.7143	,	0.8736	)
			(	5.9184	,	0.8843	)
			(	6.1224	,	0.8942	)
			(	6.3265	,	0.9032	)
			(	6.5306	,	0.9115	)
			(	6.7347	,	0.9191	)
			(	6.9388	,	0.9261	)
			(	7.1429	,	0.9325	)
			(	7.3469	,	0.9384	)
			(	7.5510	,	0.9437	)
			(	7.7551	,	0.9486	)
			(	7.9592	,	0.9531	)
			(	8.1633	,	0.9572	)
			(	8.3673	,	0.9610	)
			(	8.5714	,	0.9644	)
			(	8.7755	,	0.9676	)
			(	8.9796	,	0.9704	)
			(	9.1837	,	0.9731	)
			(	9.3878	,	0.9754	)
			(	9.5918	,	0.9776	)
			(	9.7959	,	0.9796	)
			(	10.0000	,	0.9814	)
		};
		
		\addlegendentry{$k=3$}

		\addplot [
		smooth,
		domain=0:10,
		samples=100,
		color=black3,
		line width=0.3mm
		] coordinates {
			(	0	,	0	)
			(	0.2041	,	0.0009	)
			(	0.4082	,	0.0049	)
			(	0.6122	,	0.0126	)
			(	0.8163	,	0.0240	)
			(	1.0204	,	0.0391	)
			(	1.2245	,	0.0575	)
			(	1.4286	,	0.0788	)
			(	1.6327	,	0.1027	)
			(	1.8367	,	0.1288	)
			(	2.0408	,	0.1565	)
			(	2.2449	,	0.1857	)
			(	2.4490	,	0.2158	)
			(	2.6531	,	0.2467	)
			(	2.8571	,	0.2780	)
			(	3.0612	,	0.3095	)
			(	3.2653	,	0.3408	)
			(	3.4694	,	0.3720	)
			(	3.6735	,	0.4027	)
			(	3.8776	,	0.4328	)
			(	4.0816	,	0.4623	)
			(	4.2857	,	0.4909	)
			(	4.4898	,	0.5188	)
			(	4.6939	,	0.5456	)
			(	4.8980	,	0.5715	)
			(	5.1020	,	0.5964	)
			(	5.3061	,	0.6203	)
			(	5.5102	,	0.6432	)
			(	5.7143	,	0.6650	)
			(	5.9184	,	0.6858	)
			(	6.1224	,	0.7055	)
			(	6.3265	,	0.7243	)
			(	6.5306	,	0.7420	)
			(	6.7347	,	0.7589	)
			(	6.9388	,	0.7748	)
			(	7.1429	,	0.7898	)
			(	7.3469	,	0.8039	)
			(	7.5510	,	0.8172	)
			(	7.7551	,	0.8297	)
			(	7.9592	,	0.8415	)
			(	8.1633	,	0.8525	)
			(	8.3673	,	0.8629	)
			(	8.5714	,	0.8726	)
			(	8.7755	,	0.8816	)
			(	8.9796	,	0.8901	)
			(	9.1837	,	0.8980	)
			(	9.3878	,	0.9054	)
			(	9.5918	,	0.9123	)
			(	9.7959	,	0.9188	)
			(	10.0000	,	0.9248	)
		};
	
		\addlegendentry{$k=5$}
		
		\addplot [draw=none, forget plot] coordinates {(1, 0.5)};
	
	\end{axis}
	
\end{tikzpicture}

%		\caption{grafici della funzione di densità e della funzione di ripartizione per la distribuzione chi-quadro.}

	\end{figure}

	\begin{tabular*}{1\textwidth}{l l l}
		\textbf{Supporto:} & $[0,+\infty)$ & \CS{0.40} \\ \hline
		\textbf{Funzione di densità:} & $f(x)= \dfrac{1}{2^{\frac{k}{2}} \Gamma \left(\frac{k}{2} \right) } x^{ \frac{k}{2}-1} e^{- \frac{x}{2}}\,\Ind_{(0, +\infty)}$  & \CS[0.5]{0.5}\\ \hline
		\textbf{Funzione di ripartizione:}    &  $F(t) = \dfrac{\gamma \left(\frac{k}{2}, \frac{t}{2} \right)}{\Gamma \left(\frac{k}{2}\right)} $ & \CS[0.65]{0.5}\\ \hline
		\textbf{Funzione caratteristica:} & $\varphi(u)= (1-2iu)^{-\frac{k}{2}}$ \CS[0.60]{0.40} \\ \hline
		\textbf{Valore atteso:} & $\EE[X]= k$ & \CS[0.60]{0.40} \\ \hline
		\textbf{Varianza:} & $Var(X)= 2k $ & \CS[0.60]{0.40} \\
	\end{tabular*}

\clearpage
%%%%%%%%%%%%%%%%%%%%%%%%%%%%%%%%%%%%%%%%%%%%%%%%%%

\needspace{7\baselineskip}
\subsection{$t$ di Student}

	$$t(n) = \dfrac{\Nc(0,1)}{\sqrt{\frac{\chi^2(n)}{n}}}$$

	Definita come rapporto fra una normale standard e un'espressione legata alla chi-quadro, nell'ipotesi che queste due variabili aleatorie siano indipendenti.
	Riveste un ruolo importante nella statistica, in particolare nella stima della media di una popolazione.
	Il parametro $n$ viene detto \textit{numero di gradi di libertà}.

	\begin{figure}[H]

		\centering

		\begin{tikzpicture}

	\begin{axis}[
	axis lines = middle,
	ylabel = $f(x)$,
	xlabel = $x$,
	width=0.6\textwidth,
	height=0.5\textwidth,
	%yticklabels={,,},
	%xticklabels={,,}
	]
	
		\addplot [
		smooth,
		domain=-5:5,
		samples=100,
		color=black,
		line width=0.3mm
		] coordinates {
			(	-5.0000	,	0.0122	)
			(	-4.8551	,	0.0130	)
			(	-4.7101	,	0.0137	)
			(	-4.5652	,	0.0146	)
			(	-4.4203	,	0.0155	)
			(	-4.2754	,	0.0165	)
			(	-4.1304	,	0.0176	)
			(	-3.9855	,	0.0189	)
			(	-3.8406	,	0.0202	)
			(	-3.6957	,	0.0217	)
			(	-3.5507	,	0.0234	)
			(	-3.4058	,	0.0253	)
			(	-3.2609	,	0.0274	)
			(	-3.1159	,	0.0297	)
			(	-2.9710	,	0.0324	)
			(	-2.8261	,	0.0354	)
			(	-2.6812	,	0.0389	)
			(	-2.5362	,	0.0428	)
			(	-2.3913	,	0.0474	)
			(	-2.2464	,	0.0526	)
			(	-2.1014	,	0.0588	)
			(	-1.9565	,	0.0659	)
			(	-1.8116	,	0.0743	)
			(	-1.6667	,	0.0843	)
			(	-1.5217	,	0.0960	)
			(	-1.3768	,	0.1099	)
			(	-1.2319	,	0.1264	)
			(	-1.0870	,	0.1459	)
			(	-0.9420	,	0.1686	)
			(	-0.7971	,	0.1946	)
			(	-0.6522	,	0.2233	)
			(	-0.5072	,	0.2532	)
			(	-0.3623	,	0.2814	)
			(	-0.2174	,	0.3039	)
			(	-0.0725	,	0.3166	)
			(	0.0725	,	0.3166	)
			(	0.2174	,	0.3039	)
			(	0.3623	,	0.2814	)
			(	0.5072	,	0.2532	)
			(	0.6522	,	0.2233	)
			(	0.7971	,	0.1946	)
			(	0.9420	,	0.1686	)
			(	1.0870	,	0.1459	)
			(	1.2319	,	0.1264	)
			(	1.3768	,	0.1099	)
			(	1.5217	,	0.0960	)
			(	1.6667	,	0.0843	)
			(	1.8116	,	0.0743	)
			(	1.9565	,	0.0659	)
			(	2.1014	,	0.0588	)
			(	2.2464	,	0.0526	)
			(	2.3913	,	0.0474	)
			(	2.5362	,	0.0428	)
			(	2.6812	,	0.0389	)
			(	2.8261	,	0.0354	)
			(	2.9710	,	0.0324	)
			(	3.1159	,	0.0297	)
			(	3.2609	,	0.0274	)
			(	3.4058	,	0.0253	)
			(	3.5507	,	0.0234	)
			(	3.6957	,	0.0217	)
			(	3.8406	,	0.0202	)
			(	3.9855	,	0.0189	)
			(	4.1304	,	0.0176	)
			(	4.2754	,	0.0165	)
			(	4.4203	,	0.0155	)
			(	4.5652	,	0.0146	)
			(	4.7101	,	0.0137	)
			(	4.8551	,	0.0130	)
			(	5.0000	,	0.0122	)
		};
		
%		\addlegendentry{$n=1$}
		
		\addplot [
		smooth,
		domain=-5:5,
		samples=100,
		color=black2,
		line width=0.3mm
		] coordinates {
				(	-5.0000	,	0.0042	)
				(	-4.8551	,	0.0047	)
				(	-4.7101	,	0.0052	)
				(	-4.5652	,	0.0058	)
				(	-4.4203	,	0.0065	)
				(	-4.2754	,	0.0073	)
				(	-4.1304	,	0.0082	)
				(	-3.9855	,	0.0093	)
				(	-3.8406	,	0.0105	)
				(	-3.6957	,	0.0119	)
				(	-3.5507	,	0.0136	)
				(	-3.4058	,	0.0155	)
				(	-3.2609	,	0.0178	)
				(	-3.1159	,	0.0205	)
				(	-2.9710	,	0.0236	)
				(	-2.8261	,	0.0274	)
				(	-2.6812	,	0.0319	)
				(	-2.5362	,	0.0372	)
				(	-2.3913	,	0.0435	)
				(	-2.2464	,	0.0511	)
				(	-2.1014	,	0.0601	)
				(	-1.9565	,	0.0710	)
				(	-1.8116	,	0.0838	)
				(	-1.6667	,	0.0991	)
				(	-1.5217	,	0.1171	)
				(	-1.3768	,	0.1380	)
				(	-1.2319	,	0.1621	)
				(	-1.0870	,	0.1892	)
				(	-0.9420	,	0.2189	)
				(	-0.7971	,	0.2503	)
				(	-0.6522	,	0.2819	)
				(	-0.5072	,	0.3118	)
				(	-0.3623	,	0.3374	)
				(	-0.2174	,	0.3562	)
				(	-0.0725	,	0.3663	)
				(	0.0725	,	0.3663	)
				(	0.2174	,	0.3562	)
				(	0.3623	,	0.3374	)
				(	0.5072	,	0.3118	)
				(	0.6522	,	0.2819	)
				(	0.7971	,	0.2503	)
				(	0.9420	,	0.2189	)
				(	1.0870	,	0.1892	)
				(	1.2319	,	0.1621	)
				(	1.3768	,	0.1380	)
				(	1.5217	,	0.1171	)
				(	1.6667	,	0.0991	)
				(	1.8116	,	0.0838	)
				(	1.9565	,	0.0710	)
				(	2.1014	,	0.0601	)
				(	2.2464	,	0.0511	)
				(	2.3913	,	0.0435	)
				(	2.5362	,	0.0372	)
				(	2.6812	,	0.0319	)
				(	2.8261	,	0.0274	)
				(	2.9710	,	0.0236	)
				(	3.1159	,	0.0205	)
				(	3.2609	,	0.0178	)
				(	3.4058	,	0.0155	)
				(	3.5507	,	0.0136	)
				(	3.6957	,	0.0119	)
				(	3.8406	,	0.0105	)
				(	3.9855	,	0.0093	)
				(	4.1304	,	0.0082	)
				(	4.2754	,	0.0073	)
				(	4.4203	,	0.0065	)
				(	4.5652	,	0.0058	)
				(	4.7101	,	0.0052	)
				(	4.8551	,	0.0047	)
				(	5.0000	,	0.0042	)
		};
		
%		\addlegendentry{$n=3$}
		
		\addplot [ 
		smooth,
		domain=-5:5,
		samples=100,
		color=black3,
		line width=0.3mm
		] coordinates {
			(	-5.0000	,	0.0000	)
			(	-4.8551	,	0.0000	)
			(	-4.7101	,	0.0000	)
			(	-4.5652	,	0.0000	)
			(	-4.4203	,	0.0000	)
			(	-4.2754	,	0.0000	)
			(	-4.1304	,	0.0001	)
			(	-3.9855	,	0.0001	)
			(	-3.8406	,	0.0003	)
			(	-3.6957	,	0.0004	)
			(	-3.5507	,	0.0007	)
			(	-3.4058	,	0.0012	)
			(	-3.2609	,	0.0020	)
			(	-3.1159	,	0.0031	)
			(	-2.9710	,	0.0048	)
			(	-2.8261	,	0.0074	)
			(	-2.6812	,	0.0110	)
			(	-2.5362	,	0.0160	)
			(	-2.3913	,	0.0229	)
			(	-2.2464	,	0.0320	)
			(	-2.1014	,	0.0438	)
			(	-1.9565	,	0.0588	)
			(	-1.8116	,	0.0773	)
			(	-1.6667	,	0.0995	)
			(	-1.5217	,	0.1253	)
			(	-1.3768	,	0.1546	)
			(	-1.2319	,	0.1868	)
			(	-1.0870	,	0.2210	)
			(	-0.9420	,	0.2560	)
			(	-0.7971	,	0.2904	)
			(	-0.6522	,	0.3225	)
			(	-0.5072	,	0.3508	)
			(	-0.3623	,	0.3736	)
			(	-0.2174	,	0.3896	)
			(	-0.0725	,	0.3979	)
			(	0.0725	,	0.3979	)
			(	0.2174	,	0.3896	)
			(	0.3623	,	0.3736	)
			(	0.5072	,	0.3508	)
			(	0.6522	,	0.3225	)
			(	0.7971	,	0.2904	)
			(	0.9420	,	0.2560	)
			(	1.0870	,	0.2210	)
			(	1.2319	,	0.1868	)
			(	1.3768	,	0.1546	)
			(	1.5217	,	0.1253	)
			(	1.6667	,	0.0995	)
			(	1.8116	,	0.0773	)
			(	1.9565	,	0.0588	)
			(	2.1014	,	0.0438	)
			(	2.2464	,	0.0320	)
			(	2.3913	,	0.0229	)
			(	2.5362	,	0.0160	)
			(	2.6812	,	0.0110	)
			(	2.8261	,	0.0074	)
			(	2.9710	,	0.0048	)
			(	3.1159	,	0.0031	)
			(	3.2609	,	0.0020	)
			(	3.4058	,	0.0012	)
			(	3.5507	,	0.0007	)
			(	3.6957	,	0.0004	)
			(	3.8406	,	0.0003	)
			(	3.9855	,	0.0001	)
			(	4.1304	,	0.0001	)
			(	4.2754	,	0.0000	)
			(	4.4203	,	0.0000	)
			(	4.5652	,	0.0000	)
			(	4.7101	,	0.0000	)
			(	4.8551	,	0.0000	)
			(	5.0000	,	0.0000	)
		};
		
%		\addlegendentry{$n={+\infty}$}
		
		\addplot [draw=none, forget plot] coordinates {(1, 0.5)};
	
	\end{axis}
	
\end{tikzpicture}
		\hskip 1pt
		\begin{tikzpicture}
	\begin{axis}[
	axis lines = middle,
	ylabel = $F(t)$,
	xlabel = $t$,
	width=0.6\textwidth,
	height=0.5\textwidth,
	%yticklabels={,,},
	%xticklabels={,,},
	legend style={at={(axis cs: 5,0.1)},anchor=south east, font=\tiny},
	legend cell align=left
	]
	
		\addplot [
		smooth,
		domain=-5:5,
		samples=100,
		color=black,
		line width=0.3mm
		] coordinates 	{
			(	-5.0000	,	0.0628	)
			(	-4.8551	,	0.0647	)
			(	-4.7101	,	0.0666	)
			(	-4.5652	,	0.0686	)
			(	-4.4203	,	0.0708	)
			(	-4.2754	,	0.0731	)
			(	-4.1304	,	0.0756	)
			(	-3.9855	,	0.0783	)
			(	-3.8406	,	0.0811	)
			(	-3.6957	,	0.0841	)
			(	-3.5507	,	0.0874	)
			(	-3.4058	,	0.0909	)
			(	-3.2609	,	0.0947	)
			(	-3.1159	,	0.0989	)
			(	-2.9710	,	0.1033	)
			(	-2.8261	,	0.1083	)
			(	-2.6812	,	0.1136	)
			(	-2.5362	,	0.1195	)
			(	-2.3913	,	0.1261	)
			(	-2.2464	,	0.1333	)
			(	-2.1014	,	0.1414	)
			(	-1.9565	,	0.1504	)
			(	-1.8116	,	0.1605	)
			(	-1.6667	,	0.1720	)
			(	-1.5217	,	0.1851	)
			(	-1.3768	,	0.2000	)
			(	-1.2319	,	0.2170	)
			(	-1.0870	,	0.2367	)
			(	-0.9420	,	0.2595	)
			(	-0.7971	,	0.2858	)
			(	-0.6522	,	0.3160	)
			(	-0.5072	,	0.3506	)
			(	-0.3623	,	0.3894	)
			(	-0.2174	,	0.4319	)
			(	-0.0725	,	0.4770	)
			(	0.0725	,	0.5230	)
			(	0.2174	,	0.5681	)
			(	0.3623	,	0.6106	)
			(	0.5072	,	0.6494	)
			(	0.6522	,	0.6840	)
			(	0.7971	,	0.7142	)
			(	0.9420	,	0.7405	)
			(	1.0870	,	0.7633	)
			(	1.2319	,	0.7830	)
			(	1.3768	,	0.8000	)
			(	1.5217	,	0.8149	)
			(	1.6667	,	0.8280	)
			(	1.8116	,	0.8395	)
			(	1.9565	,	0.8496	)
			(	2.1014	,	0.8586	)
			(	2.2464	,	0.8667	)
			(	2.3913	,	0.8739	)
			(	2.5362	,	0.8805	)
			(	2.6812	,	0.8864	)
			(	2.8261	,	0.8917	)
			(	2.9710	,	0.8967	)
			(	3.1159	,	0.9011	)
			(	3.2609	,	0.9053	)
			(	3.4058	,	0.9091	)
			(	3.5507	,	0.9126	)
			(	3.6957	,	0.9159	)
			(	3.8406	,	0.9189	)
			(	3.9855	,	0.9217	)
			(	4.1304	,	0.9244	)
			(	4.2754	,	0.9269	)
			(	4.4203	,	0.9292	)
			(	4.5652	,	0.9314	)
			(	4.7101	,	0.9334	)
			(	4.8551	,	0.9353	)
			(	5.0000	,	0.9372	)
		};
	
		\addlegendentry{$n=1$}
		
		\addplot [ 
		smooth,
		domain=-5:5,
		samples=100,
		color=black2,
		line width=0.3mm
		] coordinates 	{
			(	-5.0000	,	0.0077	)
			(	-4.8551	,	0.0083	)
			(	-4.7101	,	0.0091	)
			(	-4.5652	,	0.0099	)
			(	-4.4203	,	0.0107	)
			(	-4.2754	,	0.0117	)
			(	-4.1304	,	0.0129	)
			(	-3.9855	,	0.0141	)
			(	-3.8406	,	0.0156	)
			(	-3.6957	,	0.0172	)
			(	-3.5507	,	0.0190	)
			(	-3.4058	,	0.0211	)
			(	-3.2609	,	0.0236	)
			(	-3.1159	,	0.0263	)
			(	-2.9710	,	0.0295	)
			(	-2.8261	,	0.0332	)
			(	-2.6812	,	0.0375	)
			(	-2.5362	,	0.0425	)
			(	-2.3913	,	0.0483	)
			(	-2.2464	,	0.0552	)
			(	-2.1014	,	0.0632	)
			(	-1.9565	,	0.0727	)
			(	-1.8116	,	0.0839	)
			(	-1.6667	,	0.0971	)
			(	-1.5217	,	0.1127	)
			(	-1.3768	,	0.1312	)
			(	-1.2319	,	0.1529	)
			(	-1.0870	,	0.1783	)
			(	-0.9420	,	0.2078	)
			(	-0.7971	,	0.2418	)
			(	-0.6522	,	0.2804	)
			(	-0.5072	,	0.3235	)
			(	-0.3623	,	0.3706	)
			(	-0.2174	,	0.4209	)
			(	-0.0725	,	0.4734	)
			(	0.0725	,	0.5266	)
			(	0.2174	,	0.5791	)
			(	0.3623	,	0.6294	)
			(	0.5072	,	0.6765	)
			(	0.6522	,	0.7196	)
			(	0.7971	,	0.7582	)
			(	0.9420	,	0.7922	)
			(	1.0870	,	0.8217	)
			(	1.2319	,	0.8471	)
			(	1.3768	,	0.8688	)
			(	1.5217	,	0.8873	)
			(	1.6667	,	0.9029	)
			(	1.8116	,	0.9161	)
			(	1.9565	,	0.9273	)
			(	2.1014	,	0.9368	)
			(	2.2464	,	0.9448	)
			(	2.3913	,	0.9517	)
			(	2.5362	,	0.9575	)
			(	2.6812	,	0.9625	)
			(	2.8261	,	0.9668	)
			(	2.9710	,	0.9705	)
			(	3.1159	,	0.9737	)
			(	3.2609	,	0.9764	)
			(	3.4058	,	0.9789	)
			(	3.5507	,	0.9810	)
			(	3.6957	,	0.9828	)
			(	3.8406	,	0.9844	)
			(	3.9855	,	0.9859	)
			(	4.1304	,	0.9871	)
			(	4.2754	,	0.9883	)
			(	4.4203	,	0.9893	)
			(	4.5652	,	0.9901	)
			(	4.7101	,	0.9909	)
			(	4.8551	,	0.9917	)
			(	5.0000	,	0.9923	)
		};
	
		\addlegendentry{$n=3$}
		
		\addplot [
		smooth,
		domain=-5:5,
		samples=100,
		color=black3,
		line width=0.3mm
		] coordinates 	{
			(	-5.0000	,	0.0000	)
			(	-4.8551	,	0.0000	)
			(	-4.7101	,	0.0000	)
			(	-4.5652	,	0.0000	)
			(	-4.4203	,	0.0000	)
			(	-4.2754	,	0.0000	)
			(	-4.1304	,	0.0000	)
			(	-3.9855	,	0.0000	)
			(	-3.8406	,	0.0001	)
			(	-3.6957	,	0.0001	)
			(	-3.5507	,	0.0002	)
			(	-3.4058	,	0.0003	)
			(	-3.2609	,	0.0006	)
			(	-3.1159	,	0.0009	)
			(	-2.9710	,	0.0015	)
			(	-2.8261	,	0.0024	)
			(	-2.6812	,	0.0037	)
			(	-2.5362	,	0.0056	)
			(	-2.3913	,	0.0084	)
			(	-2.2464	,	0.0123	)
			(	-2.1014	,	0.0178	)
			(	-1.9565	,	0.0252	)
			(	-1.8116	,	0.0350	)
			(	-1.6667	,	0.0478	)
			(	-1.5217	,	0.0640	)
			(	-1.3768	,	0.0843	)
			(	-1.2319	,	0.1090	)
			(	-1.0870	,	0.1385	)
			(	-0.9420	,	0.1731	)
			(	-0.7971	,	0.2127	)
			(	-0.6522	,	0.2571	)
			(	-0.5072	,	0.3060	)
			(	-0.3623	,	0.3586	)
			(	-0.2174	,	0.4140	)
			(	-0.0725	,	0.4711	)
			(	0.0725	,	0.5289	)
			(	0.2174	,	0.5860	)
			(	0.3623	,	0.6414	)
			(	0.5072	,	0.6940	)
			(	0.6522	,	0.7429	)
			(	0.7971	,	0.7873	)
			(	0.9420	,	0.8269	)
			(	1.0870	,	0.8615	)
			(	1.2319	,	0.8910	)
			(	1.3768	,	0.9157	)
			(	1.5217	,	0.9360	)
			(	1.6667	,	0.9522	)
			(	1.8116	,	0.9650	)
			(	1.9565	,	0.9748	)
			(	2.1014	,	0.9822	)
			(	2.2464	,	0.9877	)
			(	2.3913	,	0.9916	)
			(	2.5362	,	0.9944	)
			(	2.6812	,	0.9963	)
			(	2.8261	,	0.9976	)
			(	2.9710	,	0.9985	)
			(	3.1159	,	0.9991	)
			(	3.2609	,	0.9994	)
			(	3.4058	,	0.9997	)
			(	3.5507	,	0.9998	)
			(	3.6957	,	0.9999	)
			(	3.8406	,	0.9999	)
			(	3.9855	,	1.0000	)
			(	4.1304	,	1.0000	)
			(	4.2754	,	1.0000	)
			(	4.4203	,	1.0000	)
			(	4.5652	,	1.0000	)
			(	4.7101	,	1.0000	)
			(	4.8551	,	1.0000	)
			(	5.0000	,	1.0000	)
		};
	
		\addlegendentry{$n={+\infty}$}
		
		\addplot [draw=none, forget plot] coordinates {(1, 0.5)};
	
	\end{axis}
\end{tikzpicture}

%		\caption{grafici della funzione di densità e della funzione di ripartizione per la distribuzione t di Student.}

	\end{figure}

	\begin{tabular*}{1\textwidth}{l l l}
		\textbf{Supporto:} & $\RR$ & \CS{0.40} \\ \hline
		\textbf{Funzione di densità:} &  $f(x)=\dfrac{\Gamma \left(\frac{n+1}{2} \right)}{\Gamma \left(\frac{n}{2}\right)\sqrt{\pi n}} \cdot  \left(1+\frac{x^2}{n} \right)^{-\frac{n+1}{2}}$ & \CS[0.7]{0.5} \\ \hline
		\textbf{Valore atteso:} & $\EE[X]=0$ se $n>1$ \text{ oppure indefinito}& \CS[0.60]{0.40} \\ \hline
		\textbf{Varianza:} & $Var(X)=\dfrac{n}{n-2}$ se $n>2$ \text{ oppure indefinita} & \CS[0.60]{0.40}\\
	\end{tabular*}

\clearpage
%%%%%%%%%%%%%%%%%%%%%%%%%%%%%%%%%%%%%%%%%%%%%%%%%%

\needspace{7\baselineskip}
\subsection{Weibull}

	$$W(\lambda, k)$$

	Qui trovate rappresentata la più comune parametrizzazione: $\lambda>0$, $k>0$.
	
		\begin{figure}[H] 	%%%  k/(lambda^k)*x^(k-1)*e^(-(x/lambda)^k)  %%% e^(-(x/lambda)^k)
		
		\centering
		
		\begin{tikzpicture}
		
		\begin{axis}[
		axis lines = middle,
		ylabel = $f(x)$,
		xlabel = $x$,
		width=0.5\textwidth,
		height=0.5\textwidth,
		]
		
		\addplot [
		domain=0:5,
		samples=100,
		color=black,
		line width=0.3mm
		] { %e^(-1)
			1/(1^1)*x^(1-1)*e^(-(x/1)^1)
		};
		
		%				\addlegendentry{$ k=1, \lambda = 1$}
		
		\addplot [
		domain=0.1:5,
		samples=200,
		color=black2,
		line width=0.3mm
		] {
			2/(5^0.5)*x^(0.5-1)*e^(-(x/5)^0.5)
		};
		
		%				\addlegendentry{$ k=0.5, \lambda = 5$}
		
		\addplot [
		domain=0:5,
		samples=100,
		color=black3,
		line width=0.3mm
		] {
			5/(2^5)*x^(5-1)*e^(-(x/2)^5)
		};
		
		%				\addlegendentry{$ k=5, \lambda = 2$}
		
		\end{axis}
		
		\end{tikzpicture}
		\hskip 1pt
		\begin{tikzpicture}
		
		\begin{axis}[
		axis lines = middle,
		ylabel = $F(t)$,
		xlabel = $t$,
		width=0.5\textwidth,
		height=0.5\textwidth,
		%yticklabels={,,},
		%xticklabels={,,},
		%axis equal=true,
		legend style={at={(axis cs: 5,0.1)},anchor=south east, font=\tiny},
		legend cell align=left
		]
		
		\addplot [
		domain=0:5,
		samples=100,
		color=black,
		line width=0.3mm
		] { %e^(-1)
			1-e^(-(x/1)^1)
		};
		
		\addlegendentry{$\lambda = 1,  k=1$}
		
		
		\addplot [
		domain=0:5,
		samples=100,
		color=black2,
		line width=0.3mm
		] {
			1-e^(-(x/5)^0.5)
		};
		
		\addlegendentry{$ \lambda=5, k=0.5$}
		
		\addplot [
		domain=0:5,
		samples=100,
		color=black3,
		line width=0.3mm
		] {
			1-e^(-(x/2)^5)
		};
		
		\addlegendentry{$ \lambda=2, k=5$}
		
		\addplot [draw=none, forget plot] coordinates {(1, 1.01)};
		
		\end{axis}
		
		\end{tikzpicture}
		
		%		\caption{grafici della funzione di densità e della funzione di ripartizione per la distribuzione di Weibull.}
		
	\end{figure}
	
	\begin{tabular*}{1\textwidth}{l l l}
		\textbf{Supporto:} & $[0, +\infty]$ & \CS{0.40} \\ \hline
		\textbf{Funzione di densità:} & $f(x)= \dfrac{k}{\lambda^k} x^{k-1} e^{- (x/\lambda)^k }\,\Ind_{(0, +\infty)}(x)$ & \CS[0.60]{0.40}\\ \hline
		\textbf{Funzione di ripartizione:}    &  $F(t)= 1-e^{- (\frac{t}{\lambda})^k}$ \CS[0.60]{0.40}\\ \hline
		\textbf{Funzione caratteristica:} & $\varphi(u) = \sum_{n=0}^{+\infty} \dfrac{(iu)^n\lambda^n}{n!}\Gamma \left(1+\dfrac{n}{k}  \right)$ &  \CS[0.60]{0.40}\\ \hline
		\textbf{Valore atteso:} & $\EE[X]= \frac{\lambda}{k} \Gamma( \frac{1}{k})$ &  \CS[0.60]{0.40}\\ \hline
		\textbf{Varianza:} & $Var(X)= \dfrac{\lambda^2}{k^2} \left[ 2k \Gamma \left(\frac{2}{k} \right) - \Gamma^2 \left(\frac{1}{k} \right) \right] $ &  \CS[0.60]{0.40}\\
	\end{tabular*}

	È spesso impiegata per la stimare il {tempo medio di guasto} (o MTTF, mean time to failure) di un sistema.
	L'asse delle ascisse rappresenta il tempo, o il numero di cicli a cui è sottposto il campione (per esempio cadute o oscillazioni caldo-freddo); 
	l'asse delle ordinate nella densità rappresenta la probabilità che un oggetto si guasti nel tempo, e nella funzione di ripartizione la percentuale di oggetti guasti.
	\needspace{5\baselineskip}
	Si noti che il parametro $k$, denominato \textit{tasso di guasto} o \textit{modulo di Weibull}, fornisce un'importante informazione:
	\begin{itemize}
		\item $k < 1$: la probabilità di guastarsi si riduce nel tempo, per esempio nel caso del rodaggio di un macchinario, producendo quindi un'alta  ``mortalità infantile'';
		\item $k = 1$: la probabilità di guastarsi rimane invariata, e questo è il caso particolare della distribuzione esponenziale $\Ec (\lambda)$;
		\item $k > 1$: la probabilità di guastarsi aumenta nel tempo, per esempio in un sistema dove le sollecitazioni precedenti danneggiano senza guastare.
	\end{itemize}




\clearpage
%%%%%%%%%%%%%%%%%%%%%%%%%%%%%%%%%%%%%%%%%%%%%%%%%%

\needspace{7\baselineskip}
\subsection{Cauchy}

$$\Cc(x_0, y_0)$$

Caso insolito tra le distribuzioni canoniche, in quanto una variabile con questa legge non è in $L^1$ e pertanto non ha né valore atteso né varianza.
Ha i parametri $x_0 \in \RR$, $y_0 > 0$.\\[-8pt]

\begin{figure}[H] %%% a,b  (1/pi)*(b)/((x-a)^2+b^2)      (1/pi)*arccot((a-x)/b)		(1/pi)*(pi/2-atan((a-x)/b))

	\centering

	\begin{tikzpicture}

		\begin{axis}[
		axis lines = middle,
		ylabel = $f(x)$,
		xlabel = $x$,
		width=0.6\textwidth,
		height=0.5\textwidth,
		%yticklabels={,,},
		%xticklabels={,,},
		%axis equal=true,
		legend style={at={(axis cs: -5,0.57)},anchor=north west, font=\tiny},
		legend cell align=left
		]

			\addplot [
			domain=-5:5,
			samples=100,
			color=black,
			line width=0.3mm
			] {
				(1/pi)*(1)/((x-0)^2+1^2)
			};

			\addlegendentry{$ x_0=0, y_0 = 1$}

			\addplot [
			domain=-5:5,
			samples=100,
			color=black2,
			line width=0.3mm
			] {
				(1/pi)*(0.5)/((x-2)^2+0.5^2)
			};

			\addlegendentry{$ x_0=2, y_0 = 0.5$}

			\addplot [
			domain=-5:5,
			samples=100,
			color=black3,
			line width=0.3mm
			] {
				(1/pi)*(2)/((x+1)^2+2^2)
			};

			\addlegendentry{$ x_0=-1, y_0 = 2$}

			\addplot [draw=none, forget plot] coordinates {(1, 0.5)};

		\end{axis}

	\end{tikzpicture}
	\hskip 1pt
	\begin{tikzpicture}
	\begin{axis}[
	axis lines = middle,
	ylabel = $F(t)$,
	xlabel = $t$,
	width=0.6\textwidth,
	height=0.5\textwidth,
	%yticklabels={,,},
	%xticklabels={,,}
	%legend style={at={(axis cs: -5,01)},anchor=north west, font=\tiny},
	%legend cell align=left
	]
	
		\addplot [
		smooth,
		domain=-5:5,
		samples=100,
		color=black,
		line width=0.3mm
		] coordinates 	{
			(	-5.0000	,	0.0628	)
			(	-4.7959	,	0.0654	)
			(	-4.5918	,	0.0683	)
			(	-4.3878	,	0.0713	)
			(	-4.1837	,	0.0747	)
			(	-3.9796	,	0.0784	)
			(	-3.7755	,	0.0824	)
			(	-3.5714	,	0.0869	)
			(	-3.3673	,	0.0919	)
			(	-3.1633	,	0.0975	)
			(	-2.9592	,	0.1037	)
			(	-2.7551	,	0.1108	)
			(	-2.5510	,	0.1189	)
			(	-2.3469	,	0.1282	)
			(	-2.1429	,	0.1390	)
			(	-1.9388	,	0.1516	)
			(	-1.7347	,	0.1665	)
			(	-1.5306	,	0.1842	)
			(	-1.3265	,	0.2056	)
			(	-1.1224	,	0.2317	)
			(	-0.9184	,	0.2635	)
			(	-0.7143	,	0.3026	)
			(	-0.5102	,	0.3498	)
			(	-0.3061	,	0.4054	)
			(	-0.1020	,	0.4676	)
			(	0.1020	,	0.5324	)
			(	0.3061	,	0.5946	)
			(	0.5102	,	0.6502	)
			(	0.7143	,	0.6974	)
			(	0.9184	,	0.7365	)
			(	1.1224	,	0.7683	)
			(	1.3265	,	0.7944	)
			(	1.5306	,	0.8158	)
			(	1.7347	,	0.8335	)
			(	1.9388	,	0.8484	)
			(	2.1429	,	0.8610	)
			(	2.3469	,	0.8718	)
			(	2.5510	,	0.8811	)
			(	2.7551	,	0.8892	)
			(	2.9592	,	0.8963	)
			(	3.1633	,	0.9025	)
			(	3.3673	,	0.9081	)
			(	3.5714	,	0.9131	)
			(	3.7755	,	0.9176	)
			(	3.9796	,	0.9216	)
			(	4.1837	,	0.9253	)
			(	4.3878	,	0.9287	)
			(	4.5918	,	0.9317	)
			(	4.7959	,	0.9346	)
			(	5.0000	,	0.9372	)
		};
	
%		\addlegendentry{$x_0=0, \,  y_0=1$}
		
		\addplot [ 
		smooth,
		domain=-5:5,
		samples=100,
		color=black2,
		line width=0.3mm
		] coordinates 	{
			(	-5.0000	,	0.0227	)
			(	-4.7959	,	0.0234	)
			(	-4.5918	,	0.0241	)
			(	-4.3878	,	0.0249	)
			(	-4.1837	,	0.0257	)
			(	-3.9796	,	0.0266	)
			(	-3.7755	,	0.0275	)
			(	-3.5714	,	0.0285	)
			(	-3.3673	,	0.0296	)
			(	-3.1633	,	0.0307	)
			(	-2.9592	,	0.0320	)
			(	-2.7551	,	0.0333	)
			(	-2.5510	,	0.0348	)
			(	-2.3469	,	0.0365	)
			(	-2.1429	,	0.0382	)
			(	-1.9388	,	0.0402	)
			(	-1.7347	,	0.0424	)
			(	-1.5306	,	0.0448	)
			(	-1.3265	,	0.0475	)
			(	-1.1224	,	0.0505	)
			(	-0.9184	,	0.0540	)
			(	-0.7143	,	0.0580	)
			(	-0.5102	,	0.0626	)
			(	-0.3061	,	0.0680	)
			(	-0.1020	,	0.0743	)
			(	0.1020	,	0.0820	)
			(	0.3061	,	0.0914	)
			(	0.5102	,	0.1031	)
			(	0.7143	,	0.1181	)
			(	0.9184	,	0.1378	)
			(	1.1224	,	0.1649	)
			(	1.3265	,	0.2033	)
			(	1.5306	,	0.2600	)
			(	1.7347	,	0.3447	)
			(	1.9388	,	0.4612	)
			(	2.1429	,	0.5886	)
			(	2.3469	,	0.6931	)
			(	2.5510	,	0.7654	)
			(	2.7551	,	0.8138	)
			(	2.9592	,	0.8470	)
			(	3.1633	,	0.8708	)
			(	3.3673	,	0.8884	)
			(	3.5714	,	0.9019	)
			(	3.7755	,	0.9126	)
			(	3.9796	,	0.9212	)
			(	4.1837	,	0.9284	)
			(	4.3878	,	0.9343	)
			(	4.5918	,	0.9393	)
			(	4.7959	,	0.9437	)
			(	5.0000	,	0.9474	)
		};
	
%		\addlegendentry{$x_0=2, \,  y_0=0.5$}
		
		\addplot [
		smooth,
		domain=-5:5,
		samples=100,
		color=black3,
		line width=0.3mm
		] coordinates 	{
			(	-5.0000	,	0.1476	)
			(	-4.7959	,	0.1544	)
			(	-4.5918	,	0.1617	)
			(	-4.3878	,	0.1698	)
			(	-4.1837	,	0.1785	)
			(	-3.9796	,	0.1882	)
			(	-3.7755	,	0.1988	)
			(	-3.5714	,	0.2104	)
			(	-3.3673	,	0.2233	)
			(	-3.1633	,	0.2375	)
			(	-2.9592	,	0.2533	)
			(	-2.7551	,	0.2707	)
			(	-2.5510	,	0.2900	)
			(	-2.3469	,	0.3113	)
			(	-2.1429	,	0.3348	)
			(	-1.9388	,	0.3603	)
			(	-1.7347	,	0.3879	)
			(	-1.5306	,	0.4175	)
			(	-1.3265	,	0.4485	)
			(	-1.1224	,	0.4805	)
			(	-0.9184	,	0.5130	)
			(	-0.7143	,	0.5452	)
			(	-0.5102	,	0.5764	)
			(	-0.3061	,	0.6063	)
			(	-0.1020	,	0.6343	)
			(	0.1020	,	0.6603	)
			(	0.3061	,	0.6841	)
			(	0.5102	,	0.7059	)
			(	0.7143	,	0.7256	)
			(	0.9184	,	0.7434	)
			(	1.1224	,	0.7595	)
			(	1.3265	,	0.7740	)
			(	1.5306	,	0.7871	)
			(	1.7347	,	0.7990	)
			(	1.9388	,	0.8098	)
			(	2.1429	,	0.8196	)
			(	2.3469	,	0.8286	)
			(	2.5510	,	0.8367	)
			(	2.7551	,	0.8442	)
			(	2.9592	,	0.8511	)
			(	3.1633	,	0.8574	)
			(	3.3673	,	0.8633	)
			(	3.5714	,	0.8687	)
			(	3.7755	,	0.8738	)
			(	3.9796	,	0.8784	)
			(	4.1837	,	0.8828	)
			(	4.3878	,	0.8869	)
			(	4.5918	,	0.8907	)
			(	4.7959	,	0.8942	)
			(	5.0000	,	0.8976	)
		};
	
%		\addlegendentry{$x_0=-1, \,  y_0=2$}
		
		\addplot [draw=none, forget plot] coordinates {(1, 0.5)};
	
	\end{axis}
\end{tikzpicture}

%	\caption{grafici della funzione di densità e della funzione di ripartizione per la distribuzione di Cauchy.}

\end{figure}

\begin{tabular*}{1\textwidth}{l l l}
	\textbf{Supporto:} & $\RR$ & \CS{0.40} \\ \hline  %na ceppa di minchia $[0, +\infty]$
	\textbf{Funzione di densità:} & $f(x)= \dfrac{1}{\pi} \dfrac{y_0}{(x-x_0)^2 + y_0^2}$ & \CS[0.60]{0.40}\\ \hline
%	\textbf{Funzione di ripartizione:}    &  $F(x)= \dfrac{1}{\pi} \operatorname{arccot} \left(\dfrac{x_0-x}{y_0} \right)$ & \CS[0.58]{0.38}\\ \hline  %%ToBeChanged, isn't correct
	\textbf{Funzione di ripartizione:}    &  $F(t)= \dfrac{1}{\pi} \operatorname{arctan} \left(\dfrac{t-x_0}{y_0} \right)  + \frac{1}{2}$ & \CS[0.60]{0.40}\\ \hline
	\textbf{Funzione caratteristica:} & $\varphi(u)= \exp \left\{ ix_0u-y_0|u| \right\}$\CS[0.60]{0.40}\\
\end{tabular*}

\clearpage
%%%%%%%%%%%%%%%%%%%%%%%%%%%%%%%%%%%%%%%%%%%%%%%%%%

\needspace{7\baselineskip}
\subsection{Beta}

$$B(\alpha, \beta)$$

Governa la probabilità $p$, a priori distribuita uniformemente, di un processo di Bernoulli dopo aver osservato $\alpha - 1$ successi e $\beta - 1$ fallimenti; con $\alpha >0$ e $\beta > 0$.

\begin{figure}[H]
	
	\centering
	
	\begin{tikzpicture}

	\begin{axis}[
	axis lines = middle,
	ylabel = $f(x)$,
	xlabel = $x$,
	width=0.53\textwidth,
	height=0.5\textwidth,
	%yticklabels={,,},
	%xticklabels={,,}
	]
	
		\addplot [
		smooth,
		domain=0:1,
		samples=100,
		color=black,
		line width=0.3mm
		] coordinates {
			(         0    ,0.2500)
			(0.0145 ,   0.2528)
			(0.0290 ,   0.2556)
			(0.0435  ,  0.2585)
			(0.0580   , 0.2615)
			(0.0725    ,0.2645)
			(0.0870,    0.2677)
			(0.1014  ,  0.2709)
			(0.1159   , 0.2742)
			(0.1304   , 0.2776)
			(0.1449    ,0.2811)
			(0.1594,    0.2848)
			(0.1739 ,   0.2885)
			(0.1884  ,  0.2924)
			(0.2029  ,  0.2963)
			(0.2174  ,  0.3005)
			(0.2319   , 0.3047)
			(0.2464  ,  0.3091)
			(0.2609  ,  0.3136)
			(0.2754  ,  0.3183)
			(0.2899  ,  0.3232)
			(0.3043  ,  0.3282)
			(0.3188  ,  0.3334)
			(0.3333 ,   0.3389)
			(0.3478 ,   0.3445)
			(0.3623 ,   0.3503)
			(0.3768  ,  0.3564)
			(0.3913  ,  0.3628)
			(0.4058 ,   0.3694)
			(0.4203  ,  0.3763)
			(0.4348  ,  0.3835)
			(0.4493   , 0.3911)
			(0.4638   , 0.3990)
			(0.4783  ,  0.4072)
			(0.4928  ,  0.4159)
			(0.5072  ,  0.4251)
			(0.5217  ,  0.4347)
			(0.5362  ,  0.4449)
			(0.5507  ,  0.4556)
			(0.5652  ,  0.4669)
			(0.5797 ,   0.4789)
			(0.5942 ,   0.4917)
			(0.6087  ,  0.5053)
			(0.6232  ,  0.5198)
			(0.6377  ,  0.5353)
			(0.6522  ,  0.5520)
			(0.6667  ,  0.5699)
			(0.6812  ,  0.5892)
			(0.6957 ,   0.6101)
			(0.7101  ,  0.6329)
			(0.7246 ,   0.6577)
			(0.7391  ,  0.6849)
			(0.7536  ,  0.7149)
			(0.7681  ,  0.7481)
			(0.7826  ,  0.7853)
			(0.7971  ,  0.8270)
			(0.8116  ,  0.8742)
			(0.8261  ,  0.9283)
			(0.8406  ,  0.9909)
			(0.8551  ,  1.0643)
			(0.8696  ,  1.1518)
			(0.8841  ,  1.2582)
			(0.8986  ,  1.3908)
			(0.9130  ,  1.5612)
			(0.9275  ,  1.7900)
			(0.9420 ,   2.1161)
			(0.9565 ,   2.6256)
			(0.9710  ,  3.5588)
			(0.9855  ,  5.9852)
		};
		
%		\addlegendentry{$n=1$}
		
		\addplot [
		smooth,
		domain=0:1,
		samples=100,
		color=black2,
		line width=0.3mm
		] coordinates {
			(	         0  ,  3.0000)
			(	0.0145  ,  2.9137)
			(	0.0290  ,  2.8286)
			(	0.0435  ,  2.7448)
			(	0.0580  ,  2.6623)
			(	0.0725  ,  2.5810)
			(	0.0870  ,  2.5009)
			(	0.1014   , 2.4222)
			(	0.1159   , 2.3447)
			(	0.1304 ,   2.2684)
			(	0.1449 ,   2.1934)
			(	0.1594 ,   2.1197)
			(	0.1739 ,   2.0473)
			(	0.1884 ,   1.9761)
			(	0.2029 ,   1.9061)
			(	0.2174  ,  1.8374)
			(	0.2319  ,  1.7700)
			(	0.2464 ,   1.7038)
			(	0.2609 ,   1.6389)
			(	0.2754 ,   1.5753)
			(	0.2899 ,   1.5129)
			(	0.3043 ,   1.4518)
			(	0.3188 ,   1.3919)
			(	0.3333 ,   1.3333)
			(	0.3478  ,  1.2760)
			(	0.3623  ,  1.2199)
			(	0.3768 ,   1.1651)
			(	0.3913  ,  1.1115)
			(	0.4058 ,   1.0592)
			(	0.4203 ,   1.0082)
			(	0.4348 ,   0.9584)
			(	0.4493 ,   0.9099)
			(	0.4638 ,   0.8626)
			(	0.4783 ,   0.8166)
			(	0.4928 ,   0.7719)
			(	0.5072 ,   0.7284)
			(	0.5217  ,  0.6862)
			(	0.5362 ,   0.6452)
			(	0.5507 ,   0.6055)
			(	0.5652 ,   0.5671)
			(	0.5797 ,   0.5299)
			(	0.5942 ,   0.4940)
			(	0.6087 ,   0.4594)
			(	0.6232 ,   0.4260)
			(	0.6377 ,   0.3938)
			(	0.6522 ,   0.3629)
			(	0.6667 ,   0.3333)
			(	0.6812 ,   0.3050)
			(	0.6957,    0.2779)
			(	0.7101  ,  0.2520)
			(	0.7246 ,   0.2275)
			(	0.7391  ,  0.2042)
			(	0.7536 ,   0.1821)
			(	0.7681  ,  0.1613)
			(	0.7826 ,   0.1418)
			(	0.7971  ,  0.1235)
			(	0.8116  ,  0.1065)
			(	0.8261 ,   0.0907)
			(	0.8406 ,   0.0762)
			(	0.8551  ,  0.0630)
			(	0.8696  ,  0.0510)
			(	0.8841  ,  0.0403)
			(	0.8986  ,  0.0309)
			(	0.9130  ,  0.0227)
			(	0.9275  ,  0.0158)
			(	0.9420  ,  0.0101)
			(	0.9565 ,   0.0057)
			(	0.9710  ,  0.0025)
			(	0.9855  ,  0.0006)
			(	1.0000   ,      0)

		};
		
%		\addlegendentry{$n=3$}
		
		\addplot [ 
		smooth,
		domain=0:1,
		samples=100,
		color=black3,
		line width=0.3mm
		] coordinates {
			(0.0145  ,  4.1533)
			(0.0290  ,  2.9368)
			(0.0435  ,  2.3979)
			(0.0580 ,   2.0767)
			(0.0725 ,   1.8574)
			(0.0870,    1.6956)
			(0.1014 ,   1.5698)
			(0.1159 ,   1.4684)
			(0.1304 ,   1.3844)
			(0.1449 ,   1.3134)
			(0.1594 ,   1.2523)
			(0.1739 ,   1.1990)
			(0.1884 ,   1.1519)
			(0.2029 ,   1.1100)
			(0.2174  ,  1.0724)
			(0.2319  ,  1.0383)
			(0.2464  ,  1.0073)
			(0.2609  ,  0.9789)
			(0.2754  ,  0.9528)
			(0.2899  ,  0.9287)
			(0.3043  ,  0.9063)
			(0.3188  ,  0.8855)
			(0.3333   , 0.8660)
			(0.3478   , 0.8478)
			(0.3623  ,  0.8307)
			(0.3768 ,   0.8145)
			(0.3913  ,  0.7993)
			(0.4058  ,  0.7849)
			(0.4203  ,  0.7713)
			(0.4348  ,  0.7583)
			(0.4493  ,  0.7460)
			(0.4638  ,  0.7342)
			(0.4783  ,  0.7230)
			(0.4928  ,  0.7123)
			(0.5072  ,  0.7020)
			(0.5217   , 0.6922)
			(0.5362  ,  0.6828)
			(0.5507  ,  0.6738)
			(0.5652  ,  0.6651)
			(0.5797  ,  0.6567)
			(0.5942  ,  0.6486)
			(0.6087  ,  0.6409)
			(0.6232  ,  0.6334)
			(0.6377   , 0.6261)
			(0.6522  ,  0.6191)
			(0.6667  ,  0.6124)
			(0.6812  ,  0.6058)
			(0.6957  ,  0.5995)
			(0.7101  ,  0.5933)
			(0.7246  ,  0.5874)
			(0.7391   , 0.5816)
			(0.7536  ,  0.5760)
			(0.7681  ,  0.5705)
			(0.7826  ,  0.5652)
			(0.7971  ,  0.5600)
			(0.8116  ,  0.5550)
			(0.8261  ,  0.5501)
			(0.8406  ,  0.5454)
			(0.8551  ,  0.5407)
			(0.8696  ,  0.5362)
			(0.8841   , 0.5318)
			(0.8986  ,  0.5275)
			(0.9130  ,  0.5233)
			(0.9275  ,  0.5192)
			(0.9420  ,  0.5152)
			(0.9565  ,  0.5112)
			(0.9710   , 0.5074)
			(0.9855  ,  0.5037)
			(1.0000 ,   0.5000)

		};
		
%		\addlegendentry{$n={+\infty}$}
		
		\addplot [draw=none, forget plot] coordinates {(1, 0.5)};
	
	\end{axis}
	
\end{tikzpicture}
	\hskip 1pt
	\begin{tikzpicture}
	\begin{axis}[
	axis lines = middle,
	ylabel = $F(t)$,
	xlabel = $t$,
	width=0.53\textwidth,
	height=0.5\textwidth,
	%yticklabels={,,},
	%xticklabels={,,},
	legend style={at={(axis cs: 1,0.1)},anchor=south east, font=\tiny},
	legend cell align=left
	]
	
		\addplot [
		smooth,
		domain=0:1,
		samples=100,
		color=black,
		line width=0.3mm
		] coordinates 	{
				         (0   ,      0)
				(0.0145  ,  0.0036)
				(0.0290   , 0.0073)
				(0.0435  ,  0.0111)
				(0.0580 ,   0.0148)
				(0.0725  ,  0.0186)
				(0.0870  ,  0.0225)
				(0.1014  ,  0.0264)
				(0.1159   , 0.0303)
				(0.1304   , 0.0343)
				(0.1449  ,  0.0384)
				(0.1594   , 0.0425)
				(0.1739   , 0.0466)
				(0.1884  ,  0.0509)
				(0.2029 ,   0.0551)
				(0.2174  ,  0.0594)
				(0.2319  ,  0.0638)
				(0.2464  ,  0.0683)
				(0.2609  ,  0.0728)
				(0.2754  ,  0.0774)
				(0.2899  ,  0.0820)
				(0.3043 ,   0.0867)
				(0.3188  ,  0.0915)
				(0.3333  ,  0.0964)
				(0.3478  ,  0.1013)
				(0.3623  ,  0.1064)
				(0.3768  ,  0.1115)
				(0.3913  ,  0.1167)
				(0.4058  ,  0.1220)
				(0.4203  ,  0.1274)
				(0.4348  ,  0.1329)
				(0.4493  ,  0.1385)
				(0.4638  ,  0.1443)
				(0.4783  ,  0.1501)
				(0.4928  ,  0.1561)
				(0.5072  ,  0.1622)
				(0.5362  ,  0.1748)
				(0.5507  ,  0.1813)
				(0.5652  ,  0.1880)
				(0.5797   , 0.1948)
				(0.5942  ,  0.2019)
				(0.6087   , 0.2091)
				(0.6232  ,  0.2165)
				(0.6377  ,  0.2242)
				(0.6522  ,  0.2320)
				(0.6667  ,  0.2402)
				(0.6812  ,  0.2486)
				(0.6957  ,  0.2573)
				(0.7246  ,  0.2756)
				(0.7391   , 0.2853)
				(0.7536  ,  0.2955)
				(0.7681  ,  0.3061)
				(0.7826  ,  0.3172)
				(0.7971  ,  0.3288)
				(0.8116  ,  0.3412)
				(0.8261  ,  0.3542)
				(0.8406  ,  0.3681)
				(0.8551  ,  0.3830)
				(0.8696 ,   0.3990)
				(0.8841  ,  0.4165)
				(0.8986  ,  0.4356)
				(0.9130  ,  0.4570)
				(0.9275  ,  0.4812)
				(0.9420  ,  0.5093)
				(0.9565 ,   0.5434)
				(0.9710   , 0.5874)
				(0.9855 ,   0.6530)
				(1.0000 ,   1.0000)
		};
	
		\addlegendentry{$\alpha=1,00$, $\beta=0,25$}
		
		\addplot [ 
		smooth,
		domain=0:1,
		samples=100,
		color=black2,
		line width=0.3mm
		] coordinates 	{
			        ( 0   ,      0)
			(0.0145  ,  0.0429)
			(0.0290   , 0.0845)
			(0.0435   , 0.1248)
			(0.0580  ,  0.1640)
			(0.0725  ,  0.2020)
			(0.0870  ,  0.2388)
			(0.1014   , 0.2745)
			(0.1159   , 0.3091)
			(0.1304   , 0.3425)
			(0.1449   , 0.3748)
			(0.1594   , 0.4061)
			(0.1739   , 0.4363)
			(0.1884  ,  0.4654)
			(0.2029 ,   0.4935)
			(0.2174  ,  0.5207)
			(0.2319  ,  0.5468)
			(0.2464   , 0.5720)
			(0.2609   , 0.5962)
			(0.2754   , 0.6195)
			(0.2899  ,  0.6419)
			(0.3043 ,   0.6634)
			(0.3188  ,  0.6840)
			(0.3333  ,  0.7037)
			(0.3478  ,  0.7226)
			(0.3623  ,  0.7407)
			(0.3768 ,   0.7580)
			(0.3913   , 0.7745)
			(0.4058  ,  0.7902)
			(0.4203  ,  0.8052)
			(0.4348  ,  0.8194)
			(0.4493  ,  0.8330)
			(0.4638   , 0.8458)
			(0.4783   , 0.8580)
			(0.4928  ,  0.8695)
			(0.5072  ,  0.8804)
			(0.5217   , 0.8906)
			(0.5362  ,  0.9003)
			(0.5507  ,  0.9093)
			(0.5652  ,  0.9178)
			(0.5797  ,  0.9258)
			(0.5942  ,  0.9332)
			(0.6087  ,  0.9401)
			(0.6232  ,  0.9465)
			(0.6377  ,  0.9524)
			(0.6522  ,  0.9579)
			(0.6667  ,  0.9630)
			(0.6812  ,  0.9676)
			(0.6957  ,  0.9718)
			(0.7101  ,  0.9756)
			(0.7246  ,  0.9791)
			(0.7391  ,  0.9822)
			(0.7536  ,  0.9850)
			(0.7681  ,  0.9875)
			(0.7826 ,   0.9897)
			(0.7971  ,  0.9916)
			(0.8116  ,  0.9933)
			(0.8261  ,  0.9947)
			(0.8406  ,  0.9959)
			(0.8551  ,  0.9970)
			(0.8696  ,  0.9978)
			(0.8841  ,  0.9984)
			(0.8986 ,   0.9990)
			(0.9130  ,  0.9993)
			(0.9275  ,  0.9996)
			(0.9420 ,   0.9998)
			(0.9565  ,  0.9999)
			(0.9710  ,  1.0000)
			(0.9855  ,  1.0000)
			(1.0000  ,  1.0000)
		};
	
		\addlegendentry{$\alpha = 1,00$ , $\beta = 3,00$}
		
		\addplot [
		smooth,
		domain=0:1,
		samples=100,
		color=black3,
		line width=0.3mm
		] coordinates 	{
			 (        0  ,       0)
			(0.0145  ,  0.1204)
			(0.0290 ,  0.1703)
			(0.0435 ,   0.2085)
			(0.0580 ,   0.2408)
			(0.0725  ,  0.2692)
			(0.0870 ,   0.2949)
			(0.1014 ,   0.3185)
			(0.1159 ,   0.3405)
			(0.1304 ,   0.3612)
			(0.1449 ,   0.3807)
			(0.1594 ,   0.3993)
			(0.1739 ,   0.4170)
			(0.1884 ,   0.4341)
			(0.2029 ,   0.4504)
			(0.2174 ,   0.4663)
			(0.2319 ,   0.4815)
			(0.2464 ,   0.4964)
			(0.2609  ,  0.5108)
			(0.2754  ,  0.5247)
			(0.2899 ,   0.5384)
			(0.3043 ,   0.5517)
			(0.3188 ,   0.5647)
			(0.3333  ,  0.5774)
			(0.3478  ,  0.5898)
			(0.3623  ,  0.6019)
			(0.3768  ,  0.6138)
			(0.3913 ,   0.6255)
			(0.4058 ,   0.6370)
			(0.4203  ,  0.6483)
			(0.4348  ,  0.6594)
			(0.4493  ,  0.6703)
			(0.4638  ,  0.6810)
			(0.4783  ,  0.6916)
			(0.4928  ,  0.7020)
			(0.5072 ,   0.7122)
			(0.5217 ,   0.7223)
			(0.5362  ,  0.7323)
			(0.5507  ,  0.7421)
			(0.5652 ,   0.7518)
			(0.5797  ,  0.7614)
			(0.5942  ,  0.7708)
			(0.6087 ,   0.7802)
			(0.6232  ,  0.7894)
			(0.6377  ,  0.7985)
			(0.6522  ,  0.8076)
			(0.6667  ,  0.8165)
			(0.6812  ,  0.8253)
			(0.6957  ,  0.8341)
			(0.7101  ,  0.8427)
			(0.7246  ,  0.8513)
			(0.7391  ,  0.8597)
			(0.7536  ,  0.8681)
			(0.7681  ,  0.8764)
			(0.7826  ,  0.8847)
			(0.7971  ,  0.8928)
			(0.8116  ,  0.9009)
			(0.8261  ,  0.9089)
			(0.8406  ,  0.9168)
			(0.8551  ,  0.9247)
			(0.8696  ,  0.9325)
			(0.8841  ,  0.9402)
			(0.8986  ,  0.9479)
			(0.9130  ,  0.9555)
			(0.9275  ,  0.9631)
			(0.9420  ,  0.9706)
			(0.9565  ,  0.9780)
			(0.9710 ,   0.9854)
			(0.9855 ,   0.9927)
			(1.0000  ,  1.0000)
		};
	
		\addlegendentry{$\alpha = 0,50$ , $\beta = 1,00$}
		
		\addplot [draw=none, forget plot] coordinates {(1, 0.5)};
	
	\end{axis}
\end{tikzpicture}
	
	
\end{figure}

\begin{tabular*}{1\textwidth}{l l l}
	\textbf{Supporto:} & $\left[0,1\right]$ & \CS{0.40} \\ \hline
	\textbf{Funzione di densità:} &  $f(x)=\dfrac{x^{\alpha-1}(1-x)^{\beta -1}}{B (\alpha, \beta)}$ & \CS[0.60]{0.40} \\ \hline
	\textbf{Valore atteso:} & $\EE[X]=\dfrac{\alpha}{\alpha+\beta}$ & \CS[0.60]{0.40} \\ \hline
	\textbf{Varianza:} & $Var(X)=\dfrac{\alpha\beta}{(\alpha+\beta)^2(\alpha+\beta+1)}$ & \CS[0.60]{0.40}\\
\end{tabular*}

Dove a denominatore della densità troviamo la \textit{funzione} Beta di Eulero, così definita:
$$B(x,y) = \int_0^1 t^{x-1} (1-t)^{y-1}\de t$$
\clearpage
%%%%%%%%%%%%%%%%%%%%%%%%%%%%%%%%%%%%%%%%%%%%%%%%%%

\needspace{7\baselineskip}
\subsection{Rayleigh}

$$Z(\sigma)$$

Descrive la variabile aleatoria $Z = \sqrt{X^2+Y^2}$, dove $X$ e $Y$ hanno legge $\Nc(0,\sigma^2)$; con $\sigma >0$.

\begin{figure}[H]
	
	\centering
	
	\begin{tikzpicture}

	\begin{axis}[
	axis lines = middle,
	ylabel = $f(x)$,
	xlabel = $x$,
	width=0.54\textwidth,
	height=0.5\textwidth,
	%yticklabels={,,},
	%xticklabels={,,}
	]
	
		\addplot [
		smooth,
		domain=0:7,
		samples=100,
		color=black,
		line width=0.3mm
		] coordinates {
		(         0 ,        0)
		(0.1014  ,  0.3975)
		(0.2029 ,   0.7474)
		(0.3043 ,   1.0115)
		(0.4058 ,   1.1677)
		(0.5072 ,   1.2128)
		(0.6087 ,   1.1605)
		(0.7101  ,  1.0360)
		(0.8116 ,   0.8695)
		(0.9130 ,   0.6894)
		(1.0145 ,   0.5180)
		(1.1159  ,  0.3698)
		(1.2174 ,   0.2513)
		(1.3188 ,   0.1627)
	(	1.4203 ,   0.1005)
	(	1.5217  ,  0.0593)
	(	1.6232  ,  0.0334)
	(	1.7246  ,  0.0180)
	(	1.8261 ,   0.0093)
	(	1.9275  ,  0.0046)
	(	2.0290 ,   0.0022)
	(	2.1304 ,   0.0010)
	(	2.2319  ,  0.0004)
	(	2.3333  ,  0.0002)
	(	2.4348 ,   0.0001)
	(	2.5362 ,   0.0000)
	(	2.6377  ,  0.0000)
	(	2.7391  ,  0.0000)
	(	2.8406  ,  0.0000)
	(	2.9420 ,   0.0000)
	(	3.0435  ,  0.0000)
	(	3.1449  ,  0.0000)
	(	3.2464  ,  0.0000)
	(	3.3478  ,  0.0000)
	(	3.4493  ,  0.0000)
	(	3.5507  ,  0.0000)
	(	3.6522  ,  0.0000)
	(	3.7536  ,  0.0000)
	(	3.8551  ,  0.0000)
	(	3.9565 ,   0.0000)
	(	4.0580 ,   0.0000)
	(	4.1594 ,   0.0000)
	(	4.2609 ,   0.0000)
	(	4.3623  ,  0.0000)
	(	4.4638   , 0.0000)
	(	4.5652  ,  0.0000)
	(	4.6667  ,  0.0000)
	(	4.7681  ,  0.0000)
	(	4.8696 ,   0.0000)
	(	4.9710  ,  0.0000)
	(	5.0725 ,   0.0000)
	(	5.1739 ,   0.0000)
	(	5.2754,    0.0000)
	(	5.3768  ,  0.0000)
	(	5.4783  ,  0.0000)
	(	5.5797  ,  0.0000)
	(	5.6812  ,  0.0000)
	(	5.7826 ,   0.0000)
	(	5.8841 ,   0.0000)
	(	5.9855 ,   0.0000)
	(	6.0870  ,  0.0000)
	(	6.1884  ,  0.0000)
	(	6.2899 ,   0.0000)
	(	6.3913 ,   0.0000)
	(	6.4928 ,   0.0000)
	(	6.5942 ,   0.0000)
	(	6.6957  ,  0.0000)
	(	6.7971   , 0.0000)
 	(	6.8986  ,  0.0000)
	(	7.0000  ,  0.0000)
		};
		
%		\addlegendentry{$n=1$}
		
		\addplot [
		smooth,
		domain=0:7,
		samples=100,
		color=black2,
		line width=0.3mm
		] coordinates {
			
	(		0     ,    0)
	(		0.1014 ,   0.1009)
	(		0.2029 ,   0.1988)
	(		0.3043 ,   0.2906)
	(		0.4058  ,  0.3737)
	(		0.5072   , 0.4460)
	(		0.6087 ,   0.5058)
	(		0.7101  ,  0.5519)
	(		0.8116  ,  0.5839)
	(		0.9130  ,  0.6018)
	(		1.0145 ,   0.6064)
	(		1.1159 ,   0.5987)
	(		1.2174  ,  0.5802)
	(		1.3188 ,   0.5527)
	(		1.4203 ,   0.5180)
	(		1.5217  ,  0.4781)
	(		1.6232  ,  0.4348)
	(		1.7246   , 0.3898)
	(		1.8261   , 0.3447)
	(		1.9275  ,  0.3008)
	(		2.0290 ,   0.2590)
	(		2.1304 ,   0.2202)
	(		2.2319 ,   0.1849)
	(		2.3333,    0.1534)
	(		2.4348 ,   0.1257)
	(		2.5362  ,  0.1017)
	(		2.6377  ,  0.0814)
	(		2.7391   , 0.0643)
	(		2.8406  ,  0.0503)
	(		2.9420  ,  0.0388)
	(		3.0435   , 0.0296)
	(		3.1449 ,   0.0224)
	(		3.2464 ,   0.0167)
	(		3.3478  ,  0.0123)
	(		3.4493  ,  0.0090)
	(		3.5507  ,  0.0065)
	(		3.6522  ,  0.0046)
	(		3.7536  ,  0.0033)
	(		3.8551  ,  0.0023)
	(		3.9565 ,   0.0016)
	(		4.0580 ,   0.0011)
	(		4.1594 ,   0.0007)
	(		4.2609,    0.0005)
	(		4.3623 ,   0.0003)
	(		4.4638  ,  0.0002)
	(		4.5652  ,  0.0001)
	(		4.6667  ,  0.0001)
	(		4.7681  ,  0.0001)
	(		4.8696 ,   0.0000)
	(		4.9710  ,  0.0000)
	(		5.0725 ,   0.0000)
	(		5.1739 ,   0.0000)
	(		5.2754 ,   0.0000)
	(		5.3768  ,  0.0000)
	(		5.4783  ,  0.0000)
	(		5.5797  ,  0.0000)
	(		5.6812  ,  0.0000)
	(		5.7826  ,  0.0000)
	(		5.8841   , 0.0000)
	(		5.9855 ,   0.0000)
	(		6.0870  ,  0.0000)
	(		6.1884  ,  0.0000)
	(		6.2899  ,  0.0000)
	(		6.3913  ,  0.0000)
	(		6.4928  ,  0.0000)
	(		6.5942  ,  0.0000)
	(		6.6957   , 0.0000)
	(		6.7971   , 0.0000)
	(		6.8986   , 0.0000)
	(		7.0000   , 0.0000)
		};
		
%		\addlegendentry{$n=3$}
		
		\addplot [ 
		smooth,
		domain=0:7,
		samples=100,
		color=black3,
		line width=0.3mm
		] coordinates {
	(	         0  ,       0)
	(	0.1014   , 0.0253)
	(	0.2029   , 0.0505)
	(	0.3043 ,   0.0752)
	(	0.4058 ,   0.0994)
	(	0.5072 ,   0.1228)
	(	0.6087 ,   0.1453)
	(	0.7101  ,  0.1667)
	(	0.8116  ,  0.1869)
	(	0.9130 ,   0.2057)
	(	1.0145 ,   0.2230)
	(	1.1159  ,  0.2388)
	(	1.2174  ,  0.2529)
	(	1.3188  ,  0.2653)
	(	1.4203 ,   0.2759)
	(	1.5217  ,  0.2848)
	(	1.6232 ,   0.2919)
	(	1.7246  ,  0.2973)
	(	1.8261  ,  0.3009)
	(	1.9275  ,  0.3029)
	(	2.0290 ,   0.3032)
	(	2.1304  ,  0.3020)
	(	2.2319  ,  0.2994)
	(	2.3333 ,   0.2954)
	(	2.4348 ,   0.2901)
	(	2.5362 ,   0.2837)
	(	2.6377 ,   0.2764)
	(	2.7391 ,   0.2681)
	(	2.8406 ,   0.2590)
	(	2.9420  ,  0.2493)
	(	3.0435  ,  0.2390)
	(	3.1449  ,  0.2284)
	(	3.2464  ,  0.2174)
	(	3.3478  ,  0.2062)
	(	3.4493  ,  0.1949)
	(	3.5507  ,  0.1836)
	(	3.6522  ,  0.1723)
	(	3.7536  ,  0.1613)
	(	3.8551  ,  0.1504)
	(	3.9565 ,   0.1398)
	(	4.0580 ,   0.1295)
	(	4.1594  ,  0.1196)
	(	4.2609  ,  0.1101)
	(	4.3623  ,  0.1011)
	(	4.4638  ,  0.0925)
	(	4.5652  ,  0.0843)
	(	4.6667  ,  0.0767)
	(	4.7681  ,  0.0695)
	(	4.8696 ,   0.0628)
	(	4.9710  ,  0.0566)
	(	5.0725  ,  0.0509)
	(	5.1739  ,  0.0456)
	(	5.2754  ,  0.0407)
	(	5.3768  ,  0.0362)
	(	5.4783  ,  0.0322)
	(	5.5797  ,  0.0285)
	(	5.6812 ,   0.0251)
	(	5.7826 ,   0.0221)
	(	5.8841  ,  0.0194)
	(	5.9855  ,  0.0170)
	(	6.0870  ,  0.0148)
	(	6.1884  ,  0.0129)
	(	6.2899  ,  0.0112)
	(	6.3913  ,  0.0097)
	(	6.4928  ,  0.0084)
	(	6.5942  ,  0.0072)
	(	6.6957  ,  0.0062)
	(	6.7971  ,  0.0053)
	(	6.8986  ,  0.0045)
	(	7.0000  ,  0.0038)
		};
		
%		\addlegendentry{$n={+\infty}$}
		
		\addplot [draw=none, forget plot] coordinates {(1, 0.5)};
	
	\end{axis}
	
\end{tikzpicture}
	\hskip 1pt
	\begin{tikzpicture}
	\begin{axis}[
	axis lines = middle,
	ylabel = $F(t)$,
	xlabel = $t$,
	width=0.54\textwidth,
	height=0.5\textwidth,
	%yticklabels={,,},
	%xticklabels={,,},
	legend style={at={(axis cs: 5,0.1)},anchor=south east, font=\tiny},
	legend cell align=left
	]
	
		\addplot [
		smooth,
		domain=0:7,
		%samples=100,
		color=black,
		line width=0.3mm
		] coordinates 	{
	(	         0  ,       0)
	(	0.1014  ,  0.0204)
	(	0.2029  ,  0.0790)
	(	0.3043  ,  0.1691)
	(	0.4058  ,  0.2806)
	(	0.5072  ,  0.4023)
	(	0.6087  ,  0.5234)
	(	0.7101   , 0.6353)
	(	0.8116   , 0.7322)
	(	0.9130  ,  0.8112)
	(	1.0145  ,  0.8723)
	(	1.1159  ,  0.9171)
	(	1.2174  ,  0.9484)
	(	1.3188  ,  0.9692)
	(	1.4203  ,  0.9823)
	(	1.5217   , 0.9903)
	(	1.6232  ,  0.9949)
	(	1.7246 ,   0.9974)
	(	1.8261  ,  0.9987)
	(	1.9275  ,  0.9994)
	(	2.0290  ,  0.9997)
	(	2.1304  ,  0.9999)
	(	2.2319   , 1.0000)
	(	2.3333  ,  1.0000)
	(	2.4348  ,  1.0000)
	(	2.5362  ,  1.0000)
	(	2.6377  ,  1.0000)
	(	2.7391   , 1.0000)
	(	2.8406  ,  1.0000)
	(	2.9420  ,  1.0000)
	(	3.0435 ,   1.0000)
	(	3.1449  ,  1.0000)
	(	3.2464  ,  1.0000)
	(	3.3478  ,  1.0000)
	(	3.4493  ,  1.0000)
	(	3.5507  ,  1.0000)
	(	3.6522  ,  1.0000)
	(	3.7536  ,  1.0000)
	(	3.8551  ,  1.0000)
	(	3.9565  ,  1.0000)
	(	4.0580 ,   1.0000)
	(	4.1594  ,  1.0000)
	(	4.2609  ,  1.0000)
	(	4.3623  ,  1.0000)
	(	4.4638  ,  1.0000)
	(	4.5652  ,  1.0000)
	(	4.6667  ,  1.0000)
	(	4.7681   , 1.0000)
	(	4.8696  ,  1.0000)
	(	4.9710  ,  1.0000)
	(	5.0725 ,   1.0000)
	(	5.1739 ,   1.0000)
	(	5.2754  ,  1.0000)
	(	5.3768  ,  1.0000)
	(	5.4783  ,  1.0000)
	(	5.5797 ,   1.0000)
	(	5.6812  ,  1.0000)
	(	5.7826  ,  1.0000)
	(	5.8841  ,  1.0000)
	(	5.9855  ,  1.0000)
	(	6.0870  ,  1.0000)
	(	6.1884  ,  1.0000)
	(	6.2899 ,   1.0000)
	(	6.3913 ,   1.0000)
	(	6.4928 ,   1.0000)
	(	6.5942 ,   1.0000)
	(	6.6957 ,   1.0000)
	(	6.7971  ,  1.0000)
	(	6.8986   , 1.0000)
	(	7.0000  ,  1.0000)
		};
	
		\addlegendentry{$\sigma=0,5$}
		
		\addplot [ 
		smooth,
		domain=0:7,
		%samples=100,
		color=black2,
		line width=0.3mm
		] coordinates 	{
	(		         0   ,      0)
	(		0.1014   , 0.0051)
	(		0.2029  ,  0.0204)
	(		0.3043  ,  0.0453)
	(		0.4058 ,   0.0790)
	(		0.5072  ,  0.1207)
	(		0.6087  ,  0.1691)
	(		0.7101   , 0.2229)
	(		0.8116  ,  0.2806)
	(		0.9130 ,   0.3409)
	(		1.0145 ,   0.4023)
	(		1.1159  ,  0.4635)
	(		1.2174  ,  0.5234)
	(		1.3188  ,  0.5809)
	(		1.4203 ,   0.6353)
	(		1.5217  ,  0.6858)
	(		1.6232  ,  0.7322)
	(		1.7246 ,   0.7740)
	(		1.8261  ,  0.8112)
	(		1.9275  ,  0.8440)
	(		2.0290  ,  0.8723)
	(		2.1304   , 0.8966)
	(		2.2319   , 0.9171)
	(		2.3333  ,  0.9343)
	(		2.4348  ,  0.9484)
	(		2.5362  ,  0.9599)
	(		2.6377  ,  0.9692)
	(		2.7391   , 0.9765)
	(		2.8406  ,  0.9823)
	(		2.9420 ,   0.9868)
	(		3.0435 ,   0.9903)
	(		3.1449  ,  0.9929)
	(		3.2464  ,  0.9949)
	(		3.3478  ,  0.9963)
	(		3.4493  ,  0.9974)
	(		3.5507  ,  0.9982)
	(		3.6522  ,  0.9987)
	(		3.7536  ,  0.9991)
	(		3.8551  ,  0.9994)
	(		3.9565  ,  0.9996)
	(		4.0580  ,  0.9997)
	(		4.1594  ,  0.9998)
	(		4.2609  ,  0.9999)
	(		4.3623  ,  0.9999)
	(		4.4638   , 1.0000)
	(		4.5652  ,  1.0000)
	(		4.6667 ,   1.0000)
	(		4.7681   , 1.0000)
	(		4.8696  ,  1.0000)
	(		4.9710  ,  1.0000)
	(		5.0725 ,   1.0000)
	(		5.1739  ,  1.0000)
	(		5.2754  ,  1.0000)
	(		5.3768  ,  1.0000)
	(		5.4783  ,  1.0000)
	(		5.5797 ,   1.0000)
	(		5.6812  ,  1.0000)
	(		5.7826  ,  1.0000)
	(		5.8841  ,  1.0000)
	(		5.9855 ,   1.0000)
	(		6.0870 ,   1.0000)
	(		6.1884  ,  1.0000)
	(		6.2899 ,   1.0000)
	(		6.3913  ,  1.0000)
	(		6.4928 ,   1.0000)
	(		6.5942  ,  1.0000)
	(		6.6957  ,  1.0000)
	(		6.7971  ,  1.0000)
	(		6.8986  ,  1.0000)
	(		7.0000  ,  1.0000)
		};
	
		\addlegendentry{$\sigma=1$}
		
		\addplot [
		smooth,
		domain=-5:5,
		samples=100,
		color=black3,
		line width=0.3mm
		] coordinates 	{
		(	         0  ,       0)
		(	0.1014   , 0.0013)
		(	0.2029  ,  0.0051)
		(	0.3043  ,  0.0115)
		(	0.4058  ,  0.0204)
		(	0.5072   , 0.0317)
		(	0.6087  ,  0.0453)
		(	0.7101   , 0.0611)
		(	0.8116  ,  0.0790)
		(	0.9130 ,   0.0990)
		(	1.0145 ,   0.1207)
		(	1.1159  ,  0.1442)
		(	1.2174  ,  0.1691)
		(	1.3188  ,  0.1954)
		(	1.4203  ,  0.2229)
		(	1.5217  ,  0.2513)
		(	1.6232 ,   0.2806)
		(	1.7246 ,   0.3105)
		(	1.8261  ,  0.3409)
		(	1.9275  ,  0.3715)
		(	2.0290  ,  0.4023)
		(	2.1304   , 0.4330)
		(	2.2319   , 0.4635)
		(	2.3333  ,  0.4937)
		(	2.4348  ,  0.5234)
		(	2.5362  ,  0.5525)
		(	2.6377  ,  0.5809)
		(	2.7391   , 0.6085)
		(	2.8406  ,  0.6353)
		(	2.9420  ,  0.6611)
		(	3.0435 ,   0.6858)
		(	3.1449  ,  0.7095)
		(	3.2464  ,  0.7322)
		(	3.3478  ,  0.7536)
		(	3.4493  ,  0.7740)
		(	3.5507  ,  0.7932)
		(	3.6522  ,  0.8112)
		(	3.7536  ,  0.8282)
		(	3.8551  ,  0.8440)
		(	3.9565  ,  0.8587)
		(	4.0580 ,   0.8723)
		(	4.1594  ,  0.8850)
		(	4.2609  ,  0.8966)
		(	4.3623  ,  0.9073)
		(	4.4638  ,  0.9171)
		(	4.5652  ,  0.9261)
		(	4.6667  ,  0.9343)
		(	4.7681   , 0.9417)
		(	4.8696  ,  0.9484)
		(	4.9710  ,  0.9544)
		(	5.0725 ,   0.9599)
		(	5.1739  ,  0.9648)
		(	5.2754  ,  0.9692)
		(	5.3768  ,  0.9730)
		(	5.4783  ,  0.9765)
		(	5.5797  ,  0.9796)
		(	5.6812  ,  0.9823)
		(	5.7826  ,  0.9847)
		(	5.8841  ,  0.9868)
		(	5.9855  ,  0.9886)
		(	6.0870  ,  0.9903)
		(	6.1884  ,  0.9917)
		(	6.2899  ,  0.9929)
		(	6.3913  ,  0.9939)
		(	6.4928  ,  0.9949)
		(	6.5942  ,  0.9956)
		(	6.6957  ,  0.9963)
		(	6.7971   , 0.9969)
		(	6.8986  ,  0.9974)
		(	7.0000  ,  0.9978)
		};
	
		\addlegendentry{$\sigma=2$}
		
		\addplot [draw=none, forget plot] coordinates {(1, 0.5)};
	
	\end{axis}
\end{tikzpicture}
	
	%		\caption{grafici della funzione di densità e della funzione di ripartizione per la distribuzione t di Student.}
	
\end{figure}

\begin{tabular*}{1\textwidth}{l l l}
	\textbf{Supporto:} & $[0,+\infty)$ & \CS{0.40} \\ \hline
	\textbf{Funzione di densità:} &  $f(x)=\dfrac{x}{\sigma^2} e^{-\frac{x^2}{2\sigma^2}}$ & \CS[0.6]{0.4} \\ \hline
	\textbf{Funzione di ripartizione:} &  $F(t)=1-e^{-\frac{t^2}{2\sigma^2}}$ & \CS[0.6]{0.4} \\ \hline
	\textbf{Valore atteso:} & $\EE[X]=\sigma\sqrt{\frac \pi 2}$& \CS[0.60]{0.40} \\ \hline
	\textbf{Varianza:} & $Var(X)=(2-\frac \pi 2)\sigma^2$ & \CS[0.60]{0.40}\\
\end{tabular*}


