\section{Variabili aleatorie reali continue}
Gran parte delle grandezze aleatorie del mondo reale non possono essere descritte con una probabilità concentrata in un numero discreto di punti; da qui la necessità di introdurre le \emph{variabili aleatorie continue} che descrivono questo tipo di fenomeni.
Saranno forniti criteri per stabilire se una variabile aleatoria è continua o meno (si noti fin da subito che ciò \emph{non} significa semplicemente che la funzione rappresentante la variabile è continua).
Era già stato osservato che, nel caso in cui la probabilità di una variabile sia descritta da una densità continua, il singolo punto possiede una probabilità nulla.
Sebbene non sia problematico di per sé, questo fatto causa varie complicazioni nel calcolo del valore atteso con le regole introdotte nei capitoli precedenti, che darebbero inevitabilmente risultati pari a $0$.
Occorre dunque un nuovo strumento matematico per poter integrare funzioni continue rispetto a una probabilità continua: la \emph{misura di Lebesgue}, concetto preso in prestito dalla teoria della misura e che estende la nozione di probabilità.


\subsection{Misura di Lebesgue}

\begin{defn}
  \index{Lebesgue!misura di}
  La \textbf{misura di Lebesgue} è quella funzione $\mm:\Bc \to [0, +\infty)$ tale che:
  \begin{enumerate}
    \item $\sigma$-additività: dati $B_n \in \Bc \enspace \forall n \in \NN, \; B_k \cap B_l = \varnothing \enspace \forall k \neq l$, allora:
      $$\mm\left( \bigcup\limits_{n=1}^{{+\infty}} B_n \right) = \sum\limits_{n=1}^{{+\infty}} \mm(B_n)$$
    \item $\mm((a,b]) = b-a$
  \end{enumerate}
\end{defn}
Si noti che il punto (1) non è sufficiente a qualificare la misura di Lebesgue come misura di probabilità perché non è definita sull'intervallo $[0,1]$. Il punto (2) afferma che ogni intervallo ha misura uguale alla sua lunghezza.

\medskip
\begin{teo}[unicità della misura di Lebesgue \JPTh{11.1}]
  La misura di Lebesgue è \textit{unica}.
\end{teo}

\medskip
\begin{dimo}
  \leavevmode
  Sia $a < b$, e siano $\mm^1$ e $\mm^2$ due misure di Lebesgue. Costruiamo ora due funzioni $\mm_{a,b}^1$ e $\mm_{a,b}^2$ tali che:
  $$\mm_{a,b}^k: \Bc \to [0,1], \quad \mm_{a,b}^k(A) = \frac{\mm(A \cap (a,b])}{b-a} \quad \text{per } k = 1,2$$
  Le $\mm_{a,b}^k$ sono probabilità su $(\RR, \Bc)$: infatti sono $\sigma$-additive (in quanto lo sono le $\mm^k$) e $\mm_{a,b}^k(\RR) = 1$.
  Esse avranno dunque una funzione di ripartizione ciascuna:
  $$F_{a,b}^k(x) = \mm_{a,b}^k ((-\infty, x]) = \begin{cases}0 & x < a \\ \frac{x-a}{b-a} & a < x < b \\ 1 & x > b \end{cases}$$

  \begin{figure}[H]
    \centering
    \begin{tikzpicture}
    \begin{axis}[
      axis lines = middle,
      xlabel = $x$,
      ylabel = {$F(x)$},
      width=0.7\textwidth,
      height=0.3\textwidth,
    ]

    \addplot [draw=none, forget plot] coordinates {(-1,-0.1)};
    \addplot [draw=none, forget plot] coordinates {(5,1.4)};
    \draw [line width=0.1mm, dashed] (axis cs:0,1) -- (axis cs:4,1);
    \draw [line width=0.3mm, color=lightblue] (axis cs:-3,0) -- (axis cs:1,0) -- (axis cs:4,1) -- (axis cs:5,1);
    \draw [line width=0.1mm, dashed] (axis cs:1, 0) -- (axis cs: 1, 1.2);
    \draw [line width=0.1mm, dashed] (axis cs:4, 0) -- (axis cs: 4, 1.2);
    \end{axis}
    \end{tikzpicture}
    \label{plot-ripartizione-lebesgue}
    \caption{funzione di ripartizione $F_{1,4}^k$}
  \end{figure}

  Osservando che nessuna delle due dipende da $k$, si conclude che $F_{a,b}^1 \equiv F_{a,b}^2$ e dunque che $\mm_{a,b}^1 = \mm_{a,b}^2$ perché la funzione di ripartizione caratterizza la probabilità. Dimostriamo ora che questa proprietà vale indipendentemente dagli estremi:
  \begin{align*}
    \mm^1 (A) &= \mm^1(A \cap \RR) \\
    &= \mm^1 \left(A \cap \bigcup\limits_{n \in \ZZ} (n, n+1] \right) & \text{(per definizione di $\RR$)}\\
    &= \sum\limits_{n \in \ZZ} \mm^1 \left( A \cap (n, n+1] \right) & \text{(per $\sigma$-additività)}\\
    &= \sum\limits_{n \in \ZZ} \mm_{n, n+1}^1 (A) & \text{(per definizione di $\mm_{a,b}^k$)}\\
    &= \sum\limits_{n \in \ZZ} \mm_{n, n+1}^2 (A) & \text{(le probabilità sono uguali)}\\
    &= \mm^2(A) \quad \forall A \in \Bc & \text{(ripetendo i passaggi al contrario)}
  \end{align*}
  Questo conclude la dimostrazione.
\end{dimo}

\medskip
\begin{teob}[\JPTh{11.2}]
  La misura di Lebesgue \textit{esiste}.
\end{teob}

\medskip
\needspace{3\baselineskip}
\begin{dimo}
  \leavevmode
  Si definisca
  $$F_{n, n+1} \coloneqq \begin{cases} 0 & x \leq n \\ x-n & n < x \leq n+1 \\ 1 & x > n+1, \end{cases} $$
  \begin{figure}[H]
    \centering
    \begin{tikzpicture}
    \begin{axis}[
      axis lines = middle,
      xlabel = $x$,
      ylabel = {$F(x)$},
      width=0.6\textwidth,
      height=0.3\textwidth,
    ]

    \addplot [draw=none, forget plot] coordinates {(-1,-0.1)};
    \addplot [draw=none, forget plot] coordinates {(3,1.4)};
    \draw [line width=0.1mm, dashed] (axis cs:0,1) -- (axis cs:2,1);
    \draw [line width=0.3mm, color=lightblue] (axis cs:-3,0) -- (axis cs:1,0) -- (axis cs:2,1) -- (axis cs:3,1);
    \draw [line width=0.1mm, dashed] (axis cs:1, 0) -- (axis cs: 1, 1.2);
    \draw [line width=0.1mm, dashed] (axis cs:2, 0) -- (axis cs: 2, 1.2);
    \end{axis}
    \end{tikzpicture}
    \label{plot-ripartizione-lebesgue-esistenza}
    \caption{funzione di ripartizione $F_{1,2}$}
  \end{figure}
  che è una funzione non decrescente, continua da destra e avente valori limite $F(-\infty)=0$ e $F(+\infty)=1$: è dunque una funzione di ripartizione, associata alla quale esiste una probabilità $ \mm_{n,n+1}: \Bc \to [0,1]$ (di cui però non sappiamo molto altro). \\
  Si definisca ora $\mm:\Bc \to [0,+\infty), \ \ \mm(A) \coloneqq \sum\limits_{n \in \ZZ} \mm_{n, n+1} (A)$. Notiamo che $\mm$ è $\sigma$-additiva, in quanto $\mm_{n, n+1}$ lo è e le serie a termini positivi si scambiano; inoltre:
  $$\mm((a,b]) = \sum\limits_{n \in \ZZ} \mm_{n, n+1} ((a,b]) =
  \sum\limits_{n \in \ZZ} [ F_{n,n+1}(b)-F_{n,n+1}(a) ] = b-a$$
  La tesi è dunque dimostrata.
\end{dimo}

\medskip

\subsubsection{Integrale di Lebesgue}

L'integrale di Lebesgue è compatibile con quello di Riemann, infatti è un'estensione di quest'ultimo.
  La teoria dell'integrazione per $X:(\Omega,\Ac,\PP) \to (\RR,\Bc)$ si ripete tale e quale per $h:(\RR,\Bc,\mm)\to (\RR,\Bc)$, \textit{a meno di due variazioni}:
\begin{itemize}
  \item $h$ limitata $\centernot\implies h \in L^1(\mm)$: la misura di Lebesgue ammette infatti integrali propri anche illimitati;
  \item $L^2(\mm) \centernot\subseteq L^1(\mm)$, ulteriore conseguenza della motivazione precedente.
  \end{itemize}
\medskip

\begin{defn}
  \index{Lebesgue!integrale di}
  L'\textbf{integrale di Lebesgue} è definito come:
  $$\int_{\RR} h \; \de\mm \quad \forall h \geq 0 \quad \forall h \in L^1(\mm)$$
\end{defn}
\medskip

\begin{defn}
  \index{quasi ovunque (qo)}
  Si dice che $h = \widetilde{h}$ \textbf{quasi ovunque} (\textbf{qo}) se le due funzioni appartengono alla stessa classe  di equivalenza $[h]$:
  $$[h] = \{\widetilde{h}: \RR \to \RR \text{ boreliano tale che } \mm(h \neq \widetilde{h}) = 0\}$$
\end{defn}

\begin{nb}
  $\int_{\RR} h \, \de\mm $ dipende solo da $[h]$. Questo significa che il valore dell'integrale non cambia se le funzioni sono uguali a meno di un numero discreto (finito o numerabile) di punti.
\end{nb}

\medskip
\begin{teo}
  Sia $h: [a,b] \to \RR$, limitata e Riemann-integrabile.\\*
  Allora la funzione $h$ è Lebesgue-integrabile e i due integrali hanno lo stesso valore:
  $$\int_a^b h(x) \, \dx = \int_{[a,b]} h \, \de\mm$$
\end{teo}

\medskip
\begin{ese} Alcune funzioni Lebesgue-integrabili ma non Riemann-integrabili:
  \begin{itemize}
    \item $h(x) = \dfrac{1}{x^2} \Ind_{[1, +\infty)}(x)$\\[5pt]
      Questa funzione ammette integrale di Riemann solo improprio, mentre ammette integrale di Lebesgue proprio, grazie al criterio di convergenza dominata di $h(x) = \frac{1}{x^2} \Ind_{[1, k]}(x)$
    \item $h(x) = \Ind_{\QQ \cap [0,1]}$ (detta \textit{funzione di Dirichlet}) \\*
      Questa funzione ammette solo integrale di Lebesgue, perché $h=0$ \textit{quasi ovunque}:
      $$\mm(h\neq 0) = \mm (\QQ \cap [0,1])=\sum_{q \in \QQ \cap [0,1]}\mm(\{q\})=0$$
  \end{itemize}
\end{ese}

Poiché l'integrale di Lebesgue e quello di Riemann sono interscambiabili qualora esistano entrambi, d'ora in avanti $\de m$ sarà indicato come $\dx$ e l'integrale di Lebesgue sarà indicato come $\int_\RR h \, \dx$ per chiarezza rispetto alla variabile di integrazione.

\subsection{Densità continua di probabilità}

\begin{defn}
  \index{variabile aleatoria!reale continua}
  Una probabilità $\PP$ su $(\RR, \Bc)$ ammette \textbf{densità (continua) di probabilità} se $\exists f:\RR \to [0, +\infty)$ boreliana tale che:
  $$F(x) = \PP((-\infty, x]) = \int_{-\infty}^{x} f \, \de t = \int_{\RR} f(t) \, \Ind_{(-\infty, x]}(t) \, \de t = \int_{(-\infty, x]} f \, \de t$$

  Se $\PP = P^X$ di una VAR $X$ diciamo che $X$ è una \textbf{variabile aleatoria reale (assolutamente) continua} (abbreviato in VARC) e che $f$ è densità (continua) di $X$.
\end{defn}

Quali funzioni di ripartizione ammettono densità? In questo testo saranno enunciati solo un criterio sufficiente e uno necessario, mentre il concetto di \emph{continuità assoluta} verrà introdotto nel corso di analisi reale e funzionale.
\medskip

\begin{prop}[criterio necessario]
  Se $F(x)=\int_{-\infty}^{x}f(t) \, \de t$ allora $F$ è continua.
\end{prop}

\begin{dimo}
  \leavevmode
  Si vuole capire se, per $n \to +\infty$, $x_n \to x$ implichi $F(x_n) \to F(x)$, allora:
  $$F(x_n) = \int_{-\infty}^{x_n} f(t) \de t = \int_{\RR} f(t) \Ind_{(-\infty, x_n]} \de t \underset{n \to +\infty}{\to} \int_{\RR} f(t) \Ind_{(-\infty, x]} = F(x)$$
  L'uguaglianza vale quasi ovunque per convergenza dominata, infatti:
  $$0 \leq f(t) \Ind_{(-\infty, x_n]}(t) \leq f(t) \in L^1 \qedhere$$
\end{dimo}

\medskip
\begin{prop}[criterio sufficiente]
  Data $F \in C^0 \cap C^1$ a tratti, allora $F = \int_{-\infty}^x f \de t$, $f = F'$ quasi ovunque.
\end{prop}

\subsubsection{Caratterizzazione di una densità su $\RR$}

\begin{teob}\label{teo-f-densita}
  \begin{enumerate}
    \Fixvmode
    \item $f:\RR \to \RR$ densità di probabilità su $(\RR, \Bc)$
      $\iff f$ boreliana, $f \geq 0$, $\int_{\RR} f \, \dx = 1$;
    \item Nel caso valga (1) si dice che $f$ determina $\PP$ e vale $\PP(A) = \int_{A} f \, \dx \; \forall A \in \Bc$, ovvero la classe di equivalenza $[f]$ caratterizza $\PP$;
    \item Una probabilità $\PP$ su $(\RR, \Bc)$ con densità $f$ caratterizza $[f]$, ovvero se $f$ e $\widetilde{f}$ sono densità per $\PP$ allora $\mm(f \neq \widetilde{f}) = 0$.
  \end{enumerate}
\end{teob}
\medskip
\begin{dimo}
  \Fixvmode
  \begin{enumerate}
    \item
    \begin{itemize}
      \item ($\Longrightarrow$): $f$ boreliana, $f \geq 0$ per definizione:
        $$\int_{-\infty}^{+\infty} f \dx = \int_{-\infty}^{+\infty} \lim_{t \to +\infty} \Ind_{(-\infty, t]} (x) f(x) \dx$$
        Quindi, per convergenza monotona:
        $$ = \lim_{t \to +\infty} \int_{-\infty}^{t} f(x) \dx = \lim_{t \to +\infty} F(t) = 1$$
      \item ($\Longleftarrow$): $f$ boreliana, $f \geq 0$ per definizione, $\int_{\RR} f \dx = 1$  \\*
        Definisco $F(x) = \int_{-\infty}^{x} f \de t \implies F$ continua e decrescente.\\*
        Visto che l'operatore integrale è positivo e $f \geq 0$:
        \begin{itemize}
          \item $\lim\limits_{t \to -\infty} F(t) = 0 \quad$Per convergenza dominata
          \item $\lim\limits_{t \to +\infty} F(t) = 1 \quad$Per convergenza monotona/dominata
        \end{itemize}
        Quindi sia $\PP$ la probabilità su $(\RR, \Bc)$ della funzione di ripartizione $F$
        allora $\PP$ ha densità $f$ per costruzione.
    \end{itemize}
    \medskip
    \item Sia $f$ boreliana, $f \geq 0$, $\int_{\RR} f \dx = 1$, $f$ densità di $\PP$\\*
      Sia inoltre:
      $$\widetilde{\PP}(A) = \int_{A} f \dx \quad \forall A \in \Bc \implies \widetilde{\PP}: \Bc \to [0,1], \ \widetilde{\PP}(\RR) = 1$$
      allora $\widetilde{\PP}$ è $\sigma$-additiva.
      Quindi $\widetilde{\PP}$ è una probabilità su $(\RR, \Bc), \text{ con } \widetilde{F} = F$, $\widetilde{\PP} = \PP$, ovvero:
      $$\PP(A) = \widetilde{\PP}(A) = \int_{A} f \dx$$

      Per giustificare il passaggio in cui si afferma che $\widetilde{\PP}$ è $\sigma$-additiva si può mostrare che dati $B_n$ disgiunti, ovvero:
      $$B_n \in \Bc, B_k \cap B_l = \varnothing \; \forall k \neq l$$
      allora:
      $$\widetilde{\PP} \left(\bigcup\limits_{n} B_n \right) = \bigintsss_{\bigcup\limits_{n} B_n} f(x) \dx = \bigintsss_{\RR} f(x) \Ind_{\bigcup\limits_{n} B_n}  (x) \dx$$
      Essendo i $B_n$ disgiunti la funzione indicatrice può essere spezzata in una sommatoria:
      $$= \int_{\RR} f(x) \sum\limits_{n} \Ind_{B_n} (x) \dx  = \sum\limits_{n} \int_{\RR} f(x) \Ind_{B_n} (x) \dx = \sum\limits_{n} \widetilde{\PP}(B_n)$$
      Lo scambio di integrale e sommatoria è possibile grazie al fatto che la serie è completamente a termini positivi.

      Ricapitolando $f$ determina $\PP$ e $[f]$ caratterizza $\PP$, ovvero se $\PP_1$ ha densità $f_1$, $\PP_2$ ha densità $f_2$ e:
      $$[f_1] = [f_2] \implies \PP_1 = \PP_2$$
    \medskip
    \item Sia $\PP$ una probabilità su $(\RR, \Bc)$, $f$ e $\widetilde{f}$ densità di $\PP$;
      si vuole mostrare che questo implica $\mm (f \neq \widetilde{f}) = 0$.
      Si dimostra quindi che $\mm (f < \widetilde{f}) = 0$ e $\mm (f > \widetilde{f}) = 0$.

      Dato $\varepsilon > 0$ sia:
      $$A = \left\{x: f(x)+ \varepsilon \leq \widetilde{f}(x)\right\} \subseteq \left(f < \widetilde{f} \, \right) \implies A \uparrow \left(f  < \widetilde{f} \, \right) \text{ per } \varepsilon \downarrow 0$$
      allora, passando alle probabilità:
      $$\PP(A) + \varepsilon \mm(A) = \int_{A} (f+\varepsilon) \dx \leq \int_{A}  \widetilde{f}(A) = \PP(A)$$
      Essendo $\varepsilon$ strettamente maggiore di zero:
      $$\mm(A) = 0 \ \forall \varepsilon \implies \mm(f < \widetilde{f}) = 0$$
      Ripetendo lo stesso procedimento per $f > \widetilde{f}$ si ottiene che:
      $$\mm(f \neq \widetilde{f}) = \mm(f < \widetilde{f}) + \mm(f > \widetilde{f}) = 0 \qedhere$$
  \end{enumerate}
\end{dimo}

\medskip
\begin{teo}[il ritorno della regola del valore atteso]
  \index{valore atteso!per VAR continue}
  \index{regola del valore atteso}
  Dati $\Dom$ spazio di probabilità, $X$ VAR continua di legge $P^X$ e densità $f$ e $h: \RR \to \RR$ boreliana, allora:
  \begin{enumerate}
    \item $h \in L^1 (P^X) \iff hf \in L^1(\mm)$
    \item $h$ ammette integrale rispetto a $P^X$ $\iff$ $hf$ ammette integrale rispetto a $\mm$
    \item Sotto le ipotesi di (1) e (2):
    $$\EE[h(x)] = \int_{\Omega} h(x) \, \dPP = \int_{\RR} h \, \de P^X = \int_{\RR} h f \, \dx$$
  \end{enumerate}
\end{teo}

L'ultima uguaglianza è la tesi fondamentale del teorema.\footnote{Come al solito}
\medskip

\begin{dimo}
  La dimostrazione è molto simile a quella del teorema \ref{regola-valore-atteso}, è sufficiente considerare che:
  \begin{itemize}
    \item $B \in \Bc$, $h = \Ind_{B}$ per il teorema \ref{teo-f-densita}, quindi:
      $$P^X(B) = \int_{B} f(x) \, \dx$$
    \item $h$ è semplice per linearità;
    \item quando $h \geq 0$ si sfrutta convergenza monotona;
    \item per $h$ qualsiasi si può impiegare la scomposizione in $h_+$ e $h_-$. \qedhere
  \end{itemize}
\end{dimo}
\smallskip
\begin{nb}
  Data una variabile continua $X$ non è detto che $Y=h(X)$ lo sia altrettanto. Un esempio può essere una semplice bernoulliana: $Y = \Ind_B(X)$. $Y$ può infatti assumere solo i due valori $1$ e $0$, quindi non può essere continua a meno che non sia costante.
\end{nb}

\esercitazione{8}{11.04.17}
\subsection{Considerazioni pratiche}
Dato uno spazio di probabilità $(\Omega, \Ac, \PP)$ e una VAR $X:\RR \to \RR$, $X$ è \textbf{(assolutamente) continua}:
\begin{itemize}
  \item[] $\iff P^X$ ammette densità $f$, $\enspace P^X(B) = \int_{B}f(x) \, \dx \quad \forall B \in \Bc$
  \item[] $\iff F_X(t) = \int_{-\infty}^{+\infty} f \, \dx  \enspace \forall t \in \RR$
  \item[] $\implies F_X \in C^0$, ovvero $P^X(\{x\}) = \PP(X=x) = F_X(x)-F_X(x_-) = 0$ (in quanto $F_X(x_-)=F_X(x_+)=F_X(x)$)
  \item[] $\; \Longleftarrow \; F_X \in C^0 \cap C^1$ a tratti\\[5pt]
    In quest'ultimo caso $f = \begin{cases} F_X'(x) & $dove $F_X$ è derivabile$ \\  $valore arbitrario$ & $altrimenti$ \end{cases}$
\end{itemize}
\smallskip
Se $h$ è boreliana, e se $h:\RR \to [0,+\infty]$ oppure $h:\RR \to \RR \;(h \in L^1(\PP))$, allora:
$$\EE[h(X)] = \int_{\Omega} h(X(\omega)) \, p(\de \omega) = \int_{\RR} h(x) \, P^X (\dx) =
\int_{\RR} h(x) \, f(x) \, \dx$$
Inoltre:
$$f \text{ densità } \iff \begin{cases}
  f $ boreliana$\\
  f \geq 0 \\
  \int_{-\infty}^{{+\infty}}f(x) \, \dx = 1
\end{cases}$$

\cleardoublepage
