\paragraph{Pagina 10} Nelle definizioni 1.4 e 1.5 gli indici delle unioni devono essere $i$ invece che $k$.

\paragraph{Pagina 14} Nella dimostrazione del teorema 1.3, punto 2, $B_{n+1} \geq B_n$ deve essere $B_{n+1} \supseteq B_n$ e $\PP(B)$ deve essere $\PP(B_n)$

\paragraph{Pagina 19} Nell'esempio del test medico le percentuali di falso positivo e falso negativo sono rispettivamente $5\%$ e $2\%$

\paragraph{Pagina 103} Nel punto 3 del teorema 13.1 $\PP(X \in A | Y \in B)$ deve essere $\PP(X \in A, Y \in B)$

\paragraph{Pagina 132} Alla penultima riga dell'esempio manca una n, $\sqrt n \lambda \frac{-1} n \xrightarrow{n} 0$ deve essere $\sqrt n \lambda \frac{n-1} n \xrightarrow{n} 0$