\section*{Prefazione}
Questo libro raccoglie gli appunti del corso di Probabilità per alunni ingegneri matematici, tenuto presso il Politecnico di Milano nell'anno accademico 2016-2017 dal prof. Matteo Gregoratti, con la collaborazione del dott. Andrea Cosso per le esercitazioni.

Abbiamo cominciato a redigere le prime lezioni a computer quasi per caso, spinti prima di tutto dalla curiosità e dalla voglia di sperimentare. Ben presto ci siamo resi conto che questo metodo ci consentiva di collaborare in maniera molto semplice: ciò che sarebbe stato troppo gravoso se fatto da uno solo di noi, ora diventava fattibile.

Settimana dopo settimana, il nostro progetto ha preso forma, fino a diventare la nostra prima edizione: un piccolo manuale di istruzioni alla probabilità, da studenti per studenti. Scrivendolo abbiamo avuto l’opportunità di poter capire fino in fondo alcuni degli aspetti meno intuitivi della materia; abbiamo dunque cercato di trasmettere in maniera chiara ciò che abbiamo colto, stimolando l'intuizione di voi lettori.

Speriamo che il nostro volume vi sia di aiuto nei vostri studi almeno quanto lo è stato per noi. Realizzarlo è stata un'avventura bellissima e istruttiva: per questo vogliamo ringraziare i compagni di corso che ci hanno stimolato e motivato ad andare avanti.

Ringraziamo Carlo Vitellio, \emph{il tuo candidato per il CCS di Matematica}, cui contributo è stato di vitale importanza nella creazione del formulario e nella revisione finale dell'opera.
Ringraziamo Gregorattibot, il fedele bot che ha compilato gli appunti centinaia di volte, rendendo sempre disponibile una versione aggiornata su Dropbox.

Ringraziamo infine il prof. Matteo Gregoratti e il dott. Andrea Cosso, perché ci hanno contagiato con la loro passione, ci hanno fatto sudare e faticare, ci hanno chiesto il massimo, riuscendoci a far apprezzare la materia e con qualche stregoneria trasformando un corso da dieci crediti in uno da quindici.

\bigskip
\begin{firma}
Gli autori
\end{firma}

\clearpage
\vspace*{\stretch{1}}
{\bf Nota di redazione}\\
L'errata corrige è disponibile sul nostro sito: \url{www.fubinitonelli.it/probabilita}; per segnalare refusi e inesattezze potete contattarci a \href{mailto:info@fubinitonelli.it}{\texttt{info@fubinitonelli.it}}.

Nell'intestazione di teoremi e corollari sono stati indicati i riferimenti al libro di testo del corso: Jean Jacod, Philip Protter, \textit{Probability Essentials}, Springer-Verlag, Berlin 2004.
\vspace*{\stretch{1}}
\clearpage
