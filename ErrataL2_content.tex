\section*{Presenti nelle ristampe di marzo e aprile 2018}
\paragraph{Pagina 136} Chiariamo di seguito il testo della proposizione 10.21.

Siano $X,Y \in L^2(\PP)$ tali che $Var(X)>0$ e $Var(Y)>0$. \\
Si ha $|\rho| = 1$ se e solo se esistono $a, b \in \RR, a \neq 0,$ tali che $Y = aX + b$. In tal caso, vale:
  $$Y =
  \rho \sqrt{\dfrac{Var(Y)}{Var(X)}}X + \Ex{Y} -
  \rho \sqrt{\dfrac{Var(X)}{Var(Y)}} \Ex{X}$$

\paragraph{Pagina 192} La nota 46 legge:

Ricordiamo che ``quasi ovunque'' significa ``convergenza puntuale tranne eventualmente in insiemi di misura di Lebesgue nulla''.
