\section{Variabili aleatorie reali discrete}
In questo capitolo svilupperemo la trattazione di un'importante famiglia di variabili aleatorie reali, quelle \emph{discrete}. Esse sono un'estensione delle variabili semplici, nel senso che la loro probabilità è concentrata su singoli punti, i quali, però, possono anche essere in quantità \emph{numerabile} (e quindi non finita).

\smallskip
\begin{teo}
  \index{variabile aleatoria!reale discreta}
  Dati $\Dom$ spazio di probabilità e $X:\Omega\to\RR$ variabile aleatoria reale, allora \textit{$X$ è una variabile aleatoria reale discreta} se e solo se:
  \begin{itemize}
    \item $P^{X}$ è una probabilità discreta, cioè al più concentrata su un insieme numerabile di punti; ovvero, esiste un insieme al più numerabile $\{x_{k}\}_{k}$ di numeri reali per cui esistono coefficienti $p_{k}$, con $p_{k}\geq0$ e $\sum_{k}p_{k} = 1$, tali che $P^{X}(B)=\sum_{X_{k}\in B}p_{k}$;
    \item $F_{X}$ è costante a tratti, quindi troviamo in ogni $x_{k}$ un salto di valore pari a $p_{k}$;
    \item $\PP(X\in \{x_{k}\}_{k})=P^{X}(\{x_{k}\})=\sum_{k}p_{k}=1$;
    \item $\im(X)$ è l'insieme degli $x_{k}$ a meno degli insiemi di misura nulla;
    \item $X\aceq\widetilde{X}$, dove $\im(\widetilde{X})=\{{x}_{k}\}_{k}$ e
    $
      \widetilde{X}(\omega)=
      \begin{cases}
      X(\omega)   &\text{se } X(\omega)\in \{{x}_{k}\}_{k}\\
      x_{1}   & \text{altrimenti}
      \end{cases}
    $\\
    $x_1$ può essere arbitrario in quanto non influisce sull'immagine.
  \end{itemize}
\end{teo}

\medskip
\begin{prop}[la regola del valore atteso colpisce ancora]
  \index{valore atteso!per VAR discrete}
  \index{regola del valore atteso}
  Data $X$ VAR discreta con legge $P^{X} \leftrightarrow (\{{x}_{k}\}_{k},\{{p}_{k}\})$, allora $\forall h \geq 0$ e $\forall h\in L^{1}(P^{X})$ si ha che:
  $$\int_{\Omega} h(X)\, \dPP=\int_{\RR}h\, \dPP^{X}= \sum_{k}h(x_{k})p_{k}$$
\end{prop}
Si lascia la dimostrazione per esercizio, suggerendo di usare convergenza monotona e dominata.

\medskip
\begin{oss}
  Dalla regola del valore atteso si deducono le formule dell'attesa e della varianza per VAR discrete:
  \begin{itemize}
    \item $\mu = \EE[X]=\sum_{k}x_kp_k$ se $X \in L^1(\PP)$ oppure $X \geq 0$
    \item $Var(X)=\sum_{k}(x_k-\mu)^2p_k$ se $X \in L^2(\PP)$
  \end{itemize}
  Si presti attenzione al fatto che per la varianza è richiesta l'appartenenza a $L^2$, non solo a $L^1$.
\end{oss}

% Parte AW
\cleardoublepage
